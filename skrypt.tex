% UWAGI EDYTORSKIE
% -- jedno zdanie w jednej linii
% -- myślniki w tekście piszemy jako "--" a minus jako "-"


\documentclass{article}

\usepackage[MeX,plmath]{polski}
\usepackage[utf8]{inputenc}
\usepackage{mathtools} % nowy, ulepszony amsmath
\usepackage{amsfonts}
\usepackage{amsthm}
\usepackage{amssymb} % dodatkowe symbole
\usepackage{latexsym} % dodatkowe symbole
\usepackage{textcomp} % dodatkowe symbole
\usepackage{hyperref} % klikalne linki -- użyj \url{...}
\usepackage{array} % do skomplikowanych tabel
\usepackage{todonotes} % do notatek, nt. co jest do zrobienia; użyj \todo{...}
\usepackage{tikz} % do tworzenia grafiki
\usetikzlibrary{arrows, automata, positioning}

\title{\textbf{Logika matematyczna II} \\ -- skrypt do wykładu --}

\date{\small{Wersja z dnia: \today}}
\begin{document}

\maketitle

% Oznaczenia na zbiory liczb naturalnych, całkowitych, etc.
\newcommand{\N}{\mathbb{N}}
\newcommand{\Z}{\mathbb{Z}}
\newcommand{\Q}{\mathbb{Q}}
\newcommand{\R}{\mathbb{R}}
\newcommand{\C}{\mathbb{C}}

% Oznaczenia na ...
\newcommand{\X}{\mathcal{X}}
\newcommand{\Y}{\mathcal{Y}}

% Twierdzenia, wnioski, etc. 
\newtheorem{thm}{Thm}[section]
\theoremstyle{plain}
\newtheorem{tw}[thm]{Twierdzenie}
\newtheorem{stw}[thm]{Stwierdzenie}
\newtheorem{wn}[thm]{Wniosek}
\newtheorem{lem}[thm]{Lemat}
\newtheorem{teza}[thm]{Teza}
\newtheorem{fakt}[thm]{Fakt}

% Definicje, oznaczenia, etc.
\theoremstyle{definition}
\newtheorem{df}[thm]{Definicja}
\newtheorem{oznn}[thm]{Oznaczenia}
\newtheorem{ozn}[thm]{Oznaczenie}

% Przykłady, uwagi, etc.
\theoremstyle{remark}
\newtheorem{prz}[thm]{Przykład}
\newtheorem{uw}[thm]{Uwaga}
\newtheorem{przyp}[thm]{Przypomnienie}
\newtheorem{zd}{Zadanie}

% Oznaczenia na różne teorie, ZFC, PA, etc.
% (Skróty te działają w otoczeniu matematycznym)
\newcommand{\PA}{\mathsf{P \mkern-1.9mu A}}
\newcommand{\ZFC}{\mathsf{Z \mkern-1.8mu F \mkern-1.5mu C}}
\newcommand{\DLO}{\mathsf{D \mkern-1.5mu L \mkern-1.8mu O}}

% Skróty popularnych wyrażeń.
\newcommand{\wtw}{wtedy i tylko wtedy, gdy }
\newcommand{\fae}{następujące warunki są równoważne: }
\newcommand{\Fae}{Następujące warunki są równoważne: }

% Inne
\newcommand{\Th}{\text{Th}}


\begin{abstract}
	Notatki do wykładu \textit{Logika matematyczna II} dra Leszka Kołodziejczyka, prowadzonego w semestrze letnim roku akademickiego 2016/17.
	Spisywane przez Jędrzeja Kołodziejskiego i Bartosza Piotrowskiego.
	Uwagi o błędach są mile widziane -- proszę pisać na \texttt{bartoszpiotrowski@post.pl}
\end{abstract}


\section*{Uwagi wstępne}
Teorie aksjomatyczne są dwojakiego rodzaju:
\begin{itemize}
	\item \textit{ładne} (pod takim, czy innym względem) -- da się zrozumieć zbiory w nich definiowalne, da się zrozumieć struktury ich modeli, są rozstrzygalne; Przykłady takich teorii to: gęste liniowe porządki bez końców,  $\Th(\Q, \leq), \Th(\C,+, \bullet)$
	\item \textit{straszne}, tzw. \textit{smoki} -- powyższe nieprawdziwe - np. ZF(C), $(\N, +, \bullet, \leq)$\\
\end{itemize}
dwóm częściom odpowiadać będą dwie części kursu - do pierwszej są podręczniki












\section{Eliminacja kwantyfikatorów}
Teoria $T$ ma eliminację kwantyfikatorów (q.e.) $\iff$ dla dowolnej formuły $\psi(x_1,...x_n)$ jest $\phi(x_1,...,x_n)$ że $T\models \forall_{x_1,...,x_n[\psi(x)\iff\phi(x)]}$ (powinno być "dowodzi")\\

Uwaga: dla zdań, kiedy nie ma żadnych stałych, nie ma zdań bezkwantyfikatorowych - wtedy wrzucamy formułę z $jedną$ zmienną (alternatywnie można by dodać zmienne zdaniowe $\top, \bot$).\\

\textbf{Fakt:} dla dowolnej teorii niesprzecznej $T$ istnieje jej rozszerzenie $T^+$ z eliminacją kwantyfikatorów.\\

\textbf{Dowód:} na chama - dla każdej formuły $\phi$ dodajemy nowy symbol relacyjny $R_{\phi}$ i aksjomat $\forall_{x}[\phi(x)\iff R_{\phi}(x)]$\\


\textbf{Fakt:}\\
DLO (gęste, liniowe porządki bez końców) ma q.e.\\
- jest zupełna, bo $\om$-kategoryczna (ma tylko jeden model przeliczalny) - więc dzięki tw. Skolema-L{\"o}wenheima inaczej zdanie nierozsztrzygalne miałoby dwa modele przeliczalne, sprzeczność\\



[...]\\

Wiosek - DLO jest $o$-minimalna, tj. dla dowolnego $A\models DLO$ i $\phi(x,a)$, dow. $a\in A$ zbiór $\{x\in A\ |\ A\models \phi(x,a)\}$ (inaczej mówiąc - dowolny zbiór otwarty) jest skończoną sumą przedziałów (tu: singleton, przedział otwarty lub półprosta otwarta).
\\



Twierdzenie:\\
Dla dowolnej teorii T, równoważne są następujące warunki:\\
(i) T ma q.e.\\
(ii) dla dowolnego $\psi\in T$ i dowolnych modeli $A,B$ dla T i struktury $D$ w ich przecięciu, to dla dowolnej krotki $d\in D$, mamy $A\models\exists_{y}[\psi(d, y)]\iff B\models\exists_{y}[\psi(d, y)]$\\

Dowód:\\
(i) $\Rightarrow$ (ii) - proste, bo wywalamy kwantyfikator i wtedy nasza formuła patrzy tylko na $d$.
(ii) $\Rightarrow$ (i)\\
$[...]$\\



Definicja $A$ jest minimalne $\iff$ każdy definiowalny podzbiór $A$ jest skończony lub koskończony; jest silnie minimalna $\iff$ każde $B$ elmentarnie równoważne z $A$ jest minimalne




























\end{document}
