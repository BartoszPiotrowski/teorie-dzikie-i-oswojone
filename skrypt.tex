\documentclass{article}

\usepackage[MeX,plmath]{polski}
\usepackage[utf8]{inputenc}
\usepackage{mathtools} % nowy, ulepszony amsmath
\usepackage{amsfonts}
\usepackage{amsthm}
\usepackage{amssymb} % dodatkowe symbole
\usepackage{latexsym} % dodatkowe symbole
\usepackage{textcomp} % dodatkowe symbole
\usepackage{hyperref} % klikalne linki -- \url{...}
\usepackage{array} % do skomplikowanych tabel
\usepackage{todonotes} % do notatek; \todo{...}
\usepackage{color} % do kolorowego tekstu
\usepackage{tikz} % do tworzenia grafiki
\usepackage{xfrac} % do pisania struktur typu A/B, A/~, etc.
\usepackage{stackengine,scalerel}
\usetikzlibrary{arrows, automata, positioning}

\title{\textbf{Logika matematyczna II} \\ \Large{$\sim$ notatki z wykładu $\sim$}}

\date{\small{Wersja z dnia: \today}}

% Oznaczenia na zbiory liczb naturalnych, całkowitych, etc.
\newcommand{\N}{\mathbb{N}}
\newcommand{\Z}{\mathbb{Z}}
\newcommand{\Q}{\mathbb{Q}}
\newcommand{\R}{\mathbb{R}}
\newcommand{\A}{\mathbb{A}}
\newcommand{\B}{\mathbb{B}}
\newcommand{\C}{\mathbb{C}}
\newcommand{\D}{\mathbb{D}}
\newcommand{\E}{\mathbb{E}}
\newcommand{\F}{\mathbb{F}}
\newcommand{\K}{\mathbb{K}}

% Oznaczenia na ...
\newcommand{\X}{\mathcal{X}}
\newcommand{\Y}{\mathcal{Y}}

% Oznaczenia na funkcje
\newcommand{\id}{\text{id}}

% Twierdzenia, wnioski, etc.
\newtheorem{thm}{Thm}[section]
\theoremstyle{plain}
\newtheorem{tw}[thm]{Twierdzenie}
\newtheorem{stw}[thm]{Stwierdzenie}
\newtheorem{wn}[thm]{Wniosek}
\newtheorem{wnn}[thm]{Wnioski}
\newtheorem{lem}[thm]{Lemat}
\newtheorem{teza}[thm]{Teza}
\newtheorem{fakt}[thm]{Fakt}
\newtheorem{uw}[thm]{Uwaga}
\newtheorem{obs}[thm]{Obserwacja}

% Definicje, oznaczenia, etc.
\theoremstyle{definition}
\newtheorem{df}[thm]{Definicja}
\newtheorem{oznn}[thm]{Oznaczenia}
\newtheorem{ozn}[thm]{Oznaczenie}

% Przykłady, uwagi, etc.
\theoremstyle{remark}
\newtheorem{prz}[thm]{Przykład}
\newtheorem{przyp}[thm]{Przypomnienie}
\newtheorem{zd}{Zadanie}

%\DeclareMathOperator{\Exists}{\exists}
%\DeclareMathOperator{\Forall}{\forall}

% Oznaczenia na różne teorie, ZFC, PA, etc.
% (Skróty te działają w otoczeniu matematycznym)
\newcommand{\PA}{\mathsf{P \mkern-1.9mu A}}
\newcommand{\ZFC}{\mathsf{Z \mkern-1.6mu F \mkern-1.5mu C}}
\newcommand{\DLO}{\mathsf{D \mkern-1.5mu L \mkern-1.8mu O}}
\newcommand{\ACF}{\mathsf{A \mkern-2.0mu C \mkern-1.1mu F}}
\newcommand{\RCF}{\mathsf{R \mkern-1.7mu C \mkern-1.4mu F}}
\newcommand{\FO}{\mathsf{F \mkern-1.4mu O}}
\newcommand{\q}{\mathsf{Q}}

% Skróty popularnych wyrażeń.
\newcommand{\wtw}{wtedy i tylko wtedy, gdy }
\newcommand{\fae}{następujące warunki są równoważne: }
\newcommand{\Fae}{Następujące warunki są równoważne: }

% Inne
\newcommand{\Th}{\text{Th}}
\newcommand{\tp}{\text{tp}}
\newcommand{\Diag}{\text{Diag}}
\newcommand{\DiagEl}{\text{Diag}_{\text{el}}}

% numer Godela (rogi) \gnum{\phi}
\newbox\gnBoxA
\newdimen\gnCornerHgt
\setbox\gnBoxA=\hbox{$\ulcorner$}
\global\gnCornerHgt=\ht\gnBoxA
\newdimen\gnArgHgt
\def\gnum #1{%
\setbox\gnBoxA=\hbox{$#1$}%
\gnArgHgt=\ht\gnBoxA%
\ifnum     \gnArgHgt<\gnCornerHgt \gnArgHgt=0pt%
\else \advance \gnArgHgt by -\gnCornerHgt%
\fi \raise\gnArgHgt\hbox{$\ulcorner$} \box\gnBoxA %
\raise\gnArgHgt\hbox{$\urcorner$}}

% circled leq
\def\dclesize{\ThisStyle{\raisebox{-.1pt}{\scalebox{1.05}{$\SavedStyle\bigcirc$}}}}
\def\dcle{\ensurestackMath{\stackon[0pt]{\leq}{\dclesize}}}
\def\cleq{\def\stacktype{L}\mathbin{\scalerel*{\dcle}{\dclesize}}}

\begin{document}

\maketitle

% TODO odpowiadają +- temu, co pojawiło się na tablicy
\begin{abstract}
	Notatki z wykładu \textit{Logika matematyczna II} dra Leszka
	Kołodziejczyka, prowadzonego w semestrze letnim roku akademickiego
	2016/2017. Spisane przez Bartosza Piotrowskiego. Uwagi o błędach są
	mile widziane -- proszę pisać na \texttt{bartoszpiotrowski@post.pl}
\end{abstract}


\section*{Uwagi wstępne}
Teorie aksjomatyczne są dwojakiego rodzaju:
\begin{itemize} % może oswojone v dzikie?
	\item \textit{oswojone} (pod takim, czy innym względem) -- da się
		zrozumieć zbiory w nich definiowalne, jak również struktury ich
		modeli; są rozstrzygalne. Przykładem tego typu teorii są gęste
		liniowe porządki bez końców ($\DLO$), czyli $\Th(\Q, \leq),
		\Th(\R, +, \cdot, \leq)$, etc. Tego typu \textit{oswojone}
		teorie są przedmiotem zainteresowania teorii modeli.
	\item \textit{dzikie} (vel \textit{smoki}) -- teorie te nie spełniają
		żadnych warunków bycia \textit{oswojoną}. Przykłady: $\ZFC$,
		$\Th(\N, +, \cdot, \leq)$.
\end{itemize}
Pierwsza część niniejszego kursu będzie dotyczyła teorii \textit{oswojonych},
druga zaś -- teorii \textit{dzikich}.

Polecane podręczniki do pierwszej części to:
\begin{itemize}
	\item David Marker, \textit{Model Theory: An Introduction}, Graduate
		Texts in Mathematics, Springer, 2002,
	\item Katrin Tent, Martin Ziegler, \textit{A Course in Model Theory},
		Cambridge University Press, 2012.
\end{itemize}
% TODO oznaczenia
% co znaczy \phi(\bar{x})? \bar{x} to wszystkie zmienne wolne?

\section{Eliminacja kwantyfikatorów}
\begin{df}
	Teoria $T$ ma \textit{eliminację kwantyfikatorów} (\textit{q.e.}) \wtw dla
	dowolnej formuły $\psi(\bar{x})$ istnieje taka formuła
	bezkwantyfikatorowa $\phi(\bar{x})$, że $T \models \forall {\bar{x}}
	(\psi(\bar{x}) \iff \phi(\bar{x}))$.
\end{df}

\begin{uw}
	Jeśli w sygnaturze teorii $T$ nie występują żadne stałe, wówczas teoria
	ta nie ma zdań bezkwantyfikatorowych.  W takiej sytuacji można:
	\begin{itemize}
		\item albo dopuścić sytuację, że dla zdania $\phi$ bierzemy
			formułę bezkwantyfikatorową $\psi$ taką, że $T \vdash
			\forall x (\phi \iff \psi(x))$,
		\item albo wprowadzić zmienne zdaniowe $\top$ i $\bot$ i żądać
			zupełności teorii $T$.
	\end{itemize}
\end{uw}

\begin{prz}
	W $\Th(\R, +, \cdot, \leq)$ formuła $\phi(y, z, w) \equiv y \neq 0
	\wedge \exists x (yx^2 + zx + w = 0)$ jest równoważna formule
	bezkwantyfikatorowej $y \neq 0 \wedge z^2 - 4yw \geq 0$.
\end{prz}

Z eliminacji kwantyfikatorów typowo wynikają następujące korzyści:
\begin{itemize}
	\item zrozumienie zbiorów definiowalnych w danej teorii (no bo więcej,
		niż trzy bloki kwantyfikatorów sprawiają niemożliwym
		zrozumienie definicji),
	\item uzyskanie zupełności teorii,
	\item uzyskanie rozstrzygalności teorii.
\end{itemize}

Ale z drugiej strony trzeba uważać, bo:

\begin{stw} Dla dowolnej niesprzecznej teorii $T$ istnieje jej niesprzeczne
	rozszerzenie $T^+ \supseteq T$ z eliminacją kwantyfikatorów.
\end{stw}
\begin{proof}
	Dla każdej formuły $\phi(\bar{x})$ dodajemy nowy symbol relacyjny
	$R_{\phi}(\bar{x}).$
	\[T^+ := T \cup \{\forall \bar{x} (\phi(\bar{x}) \iff
	R_{\phi}(\bar{x}))\}.\]
\end{proof}
\ldots a nadawanie niezrozumiałym formułom nazw nie czyni ich zrozumiałymi.

\begin{tw}
	$\DLO$ (teoria gęstych liniowych porządków bez końców) posiada q.e.
\end{tw}
% nie no, to może być jaśniej i precyzyjniej napisane; jak u Markera tw. 3.1.3.
\begin{proof}
	Niech $\phi$ będzie formułą $\DLO$ w zmiennych $x_1, \ldots, x_n$.
	(Jak wiemy) $\DLO$ jest zupełna (bo ma tylko jeden, z dokładnością do
	izomorfizmu, przeliczalny model, tzn. jest $\omega$-kategoryczna, a
	mamy twierdzenie Löwenheima-Skolema).

	Zauważmy, że:
	\begin{itemize}
		\item jeśli $a_1, \ldots, a_n, b_1, \ldots b_n \in \Q$ są
			takie, że dla każdej formuły bezkwantyfikatorowej
			$\phi$ zachodzi $(\Q, \leq) \models \phi(\bar{a}) \iff
			\phi(\bar{b})$, to istnieje autoizomorfizm $h$ dla
			$(\Q, \leq)$ taki, że $h(a_i) = b_i$ dla $i \in \{1,
			\ldots, n\}$, a zatem dla każdej formuły
			$\gamma(\bar{x})$ mamy $(\Q, \leq) \models
			\gamma(\bar{a}) \iff \gamma(\bar{b})$,
		\item z dokładnością do równoważności w $\DLO$ istnieje tylko
			skończenie wiele formuł bezkwantyfikatorowych w
			zmiennych $x_1, \ldots, x_n$ -- każda jest alternatywą
			koniunkcji takich, jak na przykład ta: $x_1 = x_2
			\wedge x_1 \leq x_3 \wedge \neg x_3 \leq x_1$.
			Takie koniunkcje oznaczamy przez $\psi_1(\bar{x}),
			\psi_2(\bar{x}), \ldots$.  W takim razie
			\[(\Q, \leq) \models \forall \bar{x} (\phi(\bar{x})
			\iff \bigvee_{\mathclap{\substack{i \text{ takie, że
			istnieje } \bar{a} \in \Q^n \\ \text{ takie, że }(\Q,
			\leq) \models \phi(\bar{a}) \wedge \psi_i(\bar{a})}}}
			\psi_i(\bar{x}))\]
	\end{itemize}
	Zatem z zupełności
	\[\DLO \models \forall \bar{x} (\phi(\bar{x}) \iff
	\bigvee_{\mathclap{\substack{i \text{ takie, że istnieje } \bar{a} \in
	\Q^n \\ \text{ takie, że }(\Q, \leq) \models \phi(\bar{a}) \wedge
	\psi_i(\bar{a})}}} \psi_i(\bar{x}))\]
\end{proof}

\begin{wn}
	$\DLO$ jest \em{$o$-minimalna}, tj. dla dowolnego $A \models \DLO$ i
	dowolnej formuły $\phi(x,\bar{a})$, $\bar{a} \in A^n$, zbiór
	definiowalny $\{x \in A \colon  A \models \phi(x,\bar{a})\}$ jest
	skończoną sumą przedziałów (być może jednopunktowych).
\end{wn}
% TODO Skądinąd, DLO jest: ...

\begin{lem}
	Jeśli dla teorii $T$ i dowolnej formuły bezkwantyfikatorowej
	$\phi(\bar{x}, y)$ istnieje formuła bezkwantyfikatorowa
	$\varphi(\bar{x}, y)$ taka, że
	$T \models \exists x \phi(\bar{x}, y) \iff \varphi(\bar{x}, y)$,
	wówczas $T$ ma eliminację kwantyfikatorów.
\end{lem}
\begin{proof}
	Indukcja po złożoności formuły.
\end{proof}

\begin{tw}
Dla dowolnej teorii $T$, \fae
	\begin{enumerate}
		\item $T$ ma eliminację kwantyfikatorów,
		\item dla dowolnych dwóch struktur $\A, \B \models T$,
			dowolnego $\D$ takiego, że $\A \supseteq \D \subseteq
			\B$,  dowolnego $\bar{d} \in \D$, i dowolnej formuły
			bezkwantyfikatorowej $\psi(\bar{x}, y)$  zachodzi:
			\[\A \models \exists y \, \psi(\bar{d}, y) \iff \B
			\models \exists y \, \psi(\bar{d}, y)\]
	\end{enumerate}
\end{tw}
\begin{proof}
	($\Rightarrow$) Łatwe.

	($\Leftarrow$)
	Niech $\phi(\bar{x})$ będzie postaci $\exists y \, \varphi(\bar{x}, y)$,
	gdzie $\varphi$ jest bezkwantyfikatorowa. Na mocy lematu powyżej
	wystarczy pokazać, że z $\phi(\bar{x})$ da się wyeliminować $\exists$.

	Załóżmy bez utraty ogólności, że
	$T \not\models \forall \bar{x} \, \phi(\bar{x}) \text{ oraz }
	T \not\models \neg\forall \bar{x} \, \phi(\bar{x})$. Rozważmy zbiór
	formuł $\Gamma(\bar{x})$ zdefiniowany następująco:
	$$\Gamma(\bar{x}) = \{\gamma(\bar{x}) \text{ bezkwantyfikatorowa}
	\colon T \models \forall \bar{x} \, (\phi(\bar{x}) \Rightarrow
	\gamma(\bar{x}))\}$$
	Twierdzimy, że dla nowych stałych $\bar{d}$ zachodzi
	\[
		\tag{$\star$}
		T \cup \Gamma(\bar{d}) \models \phi(\bar{d}).
	\]
	To zakończy dowód (1), ponieważ wówczas ze zwartości dla pewnych \\
	$\gamma_1(\bar{d}), \ldots, \gamma_k(\bar{d}) \in \Gamma(\bar{d})$ mamy,
	$$T \cup \{\gamma_1(\bar{d}), \ldots, \gamma_k(\bar{d})\}
	\models \phi(\bar{d}),$$
	a więc
	$$T \models \gamma_1(\bar{d}) \wedge \ldots \wedge \gamma_k(\bar{d})
	 \iff \phi(\bar{d}),$$
	i ze ,,świeżości'' $\bar{d}$:
	$$T \models \forall \bar{x} \, (\gamma_1(\bar{x}) \wedge \ldots \wedge
	\gamma_k(\bar{x}) \iff \phi(\bar{x})).$$

	Założmy nie wprost, że ($\star$) nie zachodzi i weźmy
	$\A \models T \cup \Gamma(\bar{d}) \cup \{\neg\phi(\bar{d})\}$.
	Niech $\D$ będzie podstrukturą $\A$ wygenerowaną przez (interpretację
	stałych) $\bar{d}$. Twierdzimy, że
	$\Diag(\D) \cup T \cup \{\phi(\bar{d})\}$
	jest niesprzeczna. W przeciwnym razie dla pewnego
	$\delta(\bar{d}) \in \Diag(\D)$ teoria $\delta(\bar{d}) \cup T \cup
	\phi(\bar{d})$ jest sprzeczna i $T \cup \phi(\bar{d}) \models
	\neg \delta(\bar{d})$. Stąd $T \models \forall \bar{x} \,
	(\phi(\bar{x}) \Rightarrow \neg \delta(\bar{x}))$. Więc
	$\neg \delta(\bar{x}) \in \Gamma(\bar{x})$, ale wtedy
	$\A \models \neg \delta(\bar{d})$, i stąd
	$\D \models \neg \delta(\bar{d})$.
	Czyli $\neg\delta(\bar{d}) \in \Diag(\D)$ i sprzeczność.

	Zatem $\Diag(\D) \cup T \cup \{\phi(\bar{d})\}$ jest niesprzeczna,
	z czego wynika istnienie modelu $\B \supseteq \D$ takiego, że
	$\B \models T \cup \{\phi(\bar{d})\}$. A to jest sprzeczne z (2).
\end{proof}

\begin{df}
	Struktura $\A$ jest:
	\begin{itemize}
		\item \textit{minimalna}, jeśli każdy definiowalny podzbiór $A$
			jest skończony lub koskończony,
		\item \textit{silnie minimalna}, jeśli każda struktura $\B
			\equiv \A$ jest minimalna.
	\end{itemize}

	Teoria $T$ jest \textit{silnie minimalna} jeśli każda struktura $\B
	\models T$ jest minimalna.
\end{df}

% tu można przepisać jeszcze te zadania

\begin{df}
	$T_{\forall} = \{\phi \text{ czysto uniwersalne takie, że } T \vdash
	\phi\}$
\end{df}

\begin{stw}
	Niech $T$ to teoria. Wówczas $\A \models T_{\forall}$ \wtw istnieje $\B
	\models T$ takie, że $\A \subseteq \B$.
\end{stw}
\begin{proof}
	($\Leftarrow$) Oczywiste.

	($\Rightarrow$) Niech $\A \models T_{\forall}$. W celu uzyskania
	$\B$ wystarczy pokazać niesprzeczność teorii $\Diag(\A) \cup T$.
	Załóżmy, że jest przeciwnie -- wówczas istnieje taka fomuła
	$\delta(\bar{a})$ będąca koniunkcją zdań z $\Diag(\A)$, że $T \cup
	\{\delta(\bar{a})\}$ jest sprzeczna, czyli $T \models \neg
	\delta(\bar{a})$. Zatem $T \models \forall \bar{x} \, \neg
	\delta(\bar{x}) \in T_{\forall}$. Czyli $\A \models \forall \bar{x} \,
	\neg \delta(\bar{x})$, ale $\delta(\bar{a})$ jest koniunkcją zdań z
	$\Diag(\A)$, stąd sprzeczność.
\end{proof}

\begin{tw}
	\label{tw:qe}
	Niech $T$ będzie teorią taką, że:
	\begin{enumerate}
		\item dla dowolnych $\A, \B \models T$ takich, że $\A \subseteq
			\B$, dla dowolnego $\bar{a} \in \A$ i dowolnej formuły
			bezkwantyfikatorowej $\psi(\bar{x}, y)$, mamy:
			\[
				\B \models \exists y \, \psi(\bar{a}, y)
				\Rightarrow \A \models \exists y \, \psi(\bar{a}, y)
			\]
		\item dla każdej $\D \models T_{\forall}$ istnieje $\A \models
			T$ takie, że $\D \subseteq \A$ i każde zanurzenie $\D
			\hookrightarrow \B \models T$ rozszerza się do
			zanurzenia $\A \hookrightarrow \B$,
	\end{enumerate}
	wówczas $T$ ma eliminację kwantyfikatorów.
\end{tw}
\begin{proof}
	\color{red} TODO
\end{proof}


% WYKŁAD 2, data: 7 III 2017
\section{Ciała algebraicznie domknięte}

\begin{df}
	$\ACF$ (\textit{algebraically closed fields}) to teoria w języku $\{+,
	\cdot, -, 0, 1\}$ (gdzie ,,$-$'' jest arności $1$), wyposażona w
	następujące aksjomaty:
	\begin{itemize}
		\item aksjomaty ciała,
		\item schemat aksjomatów:
		$$\forall w_n, \ldots, w_0
			(w_n \neq 0 \Rightarrow
			\exists y (w_n y^n + \cdots + w_1y + w_0 = 0)),$$
			gdzie $n \in \N \setminus \{0\}$.
	\end{itemize}
\end{df}

\begin{tw}
	$\ACF$ posiada eliminację kwantyfikatorów.
\end{tw}
\begin{proof}
Dowodzimy następujące dwa lematy i korzystamy z Twierdzenia \ref{tw:qe}.

\begin{lem}
	$\ACF_{\forall}$ to teoria dziedzin całkowitości (tj. pierścieni bez
	dzielników zera). Ponadto zachodzi warunek (2) z Twierdzenia \ref{tw:qe}.
\end{lem}
\begin{proof}
	($\subseteq$) Każda podstruktura $\mathbb{K} \models \ACF$ jest dziedziną
	całkowitości.
\begin{przyp}[z algebry]
	~\begin{itemize}
		\item Każde ciało $\mathbb{K}$ ma swoje algebraiczne domknięcie
			$\mathbb{K}^{\text{alg}}$ (tzn. takie ciało $\mathbb{L}
			\supseteq \mathbb{K}$, że $\mathbb{L} \models \ACF$ i
			rozszerzenie $\mathbb{L} \supseteq \mathbb{K}$ jest
			algebraiczne).
		\item Algebraiczne domknięcie $\mathbb{K}^{\text{alg}}$ jest
			jedyne z dokładnością do izomorfizmu nad $\mathbb{K}$
			(tzn. jeśli $\mathbb{L}_1$ i $\mathbb{L}_2$ są
			algebraicznymi domknięciami $\mathbb{K}$, wówczas
			istnieje izomorfizm $h \colon \mathbb{L}_1
			\rightarrow \mathbb{L}_2$ taki, że $h|\mathbb{K} =
			\id(\mathbb{K})$)
	\end{itemize}
\end{przyp}
	($\supseteq$)
	Niech $\mathbb{R}$ będzie dziedziną całkowitości, a $\mathbb{F}$ niech
	będzie ciałem ułamków $\mathbb{R}$. Zauważmy, że $\mathbb{R}$ jest
	podstrukturą $\mathbb{F}^\text{alg}$.

	(zachodzenie warunku (2) z Tw. \ref{tw:qe})
	Trzymając się oznaczeń z Tw. \ref{tw:qe} twierdzimy, że dla
	$\mathcal{D} := R$, gdzie $R$ jest dziedziną całkowitości, należy wziąć
	$\mathcal{A} := F^\text{alg}$, gdzie $F$ jest ciałem ułamków $R$.
	Załóżmy, że $R$ zanuża się w $K \models \ACF$. Bez utraty ogólności,
	$R \subseteq F \subseteq K$. Weźmy $L := \text{ad}_K(F)$, % TODO notacja?
	tj. algebraiczne domknięcie $F$ wewnątrz $K$. Zauważmy, że $L$ jest
	algebraicznym domknięciem $F$. W takim razie istnieje izomorfizm
	z $\A = F^\text{alg}$ w $L$ (a zatem w $K$) będący identycznością na $F$.
\end{proof}

\begin{lem}
	Dla $\ACF$ zachodzi warunek (1) z Twierdzenia \ref{tw:qe}.
\end{lem}
\begin{proof}
	Niech $K, L \models \ACF$, $K \subseteq L$, $\bar{a} \in K$ oraz
	niech $L \models \exists y \, \psi(\bar{a}, y)$, gdzie $\psi$ jest formułą
	bezkwantyfikatorową. Bez utraty ogólności $\psi$ jest koniunkcją
	atomów i negacji atomów, czyli postaci
	\[
		\bigwedge_{i = 1}^n p_i(y) = 0 \wedge
		\bigwedge_{j = 1}^m q_j(y) \not= 0,
	\]
	gdzie $p_i, q_j \in K[x]$.
	Niech $b \in L$ takie, że $L \models \psi(\bar{a}, b)$.
	\begin{itemize}
		\item Załóżmy $n \not= 0$. Wtedy $p_1(b) = 0$, więc $b$ jest
			pierwiatkiem wielomianu z $K[x]$. % TODO x czy X?
			A skoro $K \models \ACF$, to $b \in K$.
		\item Załóżmy teraz $n = 0$, czyli że $\psi$ jest postaci
	\[
		\bigwedge_{j = 1}^m q_j(y) \not= 0.
	\]
	Ponieważ każde $q_j \not\equiv 0$, to $q_j$ ma skończenie wiele
			pierwiastków i zbiór
			$\{b \in L \colon \bigvee_{j = 1}^m q_j(b) = 0\}$
			jest skończony. $K$ jest nieskończone, więc możemy jako
			$y$ wziąć dowolny element $K \setminus S$.
	\end{itemize}
\end{proof}
Z powyższych dwóch lematów i Twierdzenia \ref{tw:qe} wynika, że $\ACF$ posiada
eliminację kwantyfikatorów.
\end{proof}

\begin{wnn}
	~\begin{itemize}
		\item $\ACF$ jest silnie minimalna, tj. w dowolnym $\mathbb{K} \models \ACF$ dowolny definiowalny podzbiór $\mathbb{K}$ jest skończony lub koskończony,
		\item jeśli $p$ jest liczbą pierwszą lub zerem, to teoria $\ACF_p := \ACF +  \{\text{char} = p\}$ jest zupełna,
		\item $\ACF$ jest rozstrzygalna.
	\end{itemize}
\end{wnn}

\begin{fakt}
	Dla każdej nieprzeliczalnej liczby porządkowej $\kappa$ teoria $\ACF_p$ jest $\kappa$-kategoryczna.
\end{fakt}
Ideą dowodu dla powyższego faktu jest to, że ciała algebraicznie domknięte tej samej charakterystyki i o tym samym stopniu przestępnym są izomorficzne.

\begin{tw}[Ax--Grothendick]
	Niech $\mathbb{K} \models \ACF$.
	Niech funkcja $f \colon K^n \rightarrow K^n$ będzie dana wielomianami, tzn. $$f(x_1, \ldots, x_n) = \langle p_1(x_1, \ldots, x_n), \ldots, p_n(x_1, \ldots, x_n)\rangle ,$$ gdzie $p_1(x_1, \ldots, x_n), \ldots, p_n(x_1, \ldots, x_n) \in K[x_1, \ldots, x_n]$. %TODO nie <> tylko cos innego.
	Wówczas jeśli $f$ jest iniekcją, to jest też surjekcją.
\end{tw}
\begin{proof}
	Rozważmy zdanie $\psi_{n,k}$ wyrażające następującą własność ciała $\K$:
	\textit{dla wszystkich $f \colon K^n \rightarrow K^n$ danych przez $p_1, p_2, \ldots, p_n$ jak wyżej, ale spełniające warunek $deg p_i \leq k$, jeśli $f$ jest iniekcją, to jest też surjekcją. }
	Zauważmy, że $\psi_{n,k}$ jest zdaniem postaci $\forall \exists$.
	Jeśli $\K$ jest skończone, to $\K \models \psi_{n,k}$ (w zbiorach skończonych iniekcje są też surjekcjami). %TODO można źle zrozumieć..
	Niech $\F^{\text{alg}}_{p}$ oznacza algebraiczne domknięcie ciała $p$-elementowego $\F^{\text{alg}}_p$.
	Zauważmy, że $$\F^{\text{alg}}_{p} = \bigcup_{m \in \N} \F_{p^m},$$ gdzie $\F_{p^m}$ to jedyne ciało o $p^m$ elementach.
	Zdania postaci $\forall \exists$ zachowują prawdziwość w sumach łańcuchów:
	jeśli $\psi$ jest takim zdaniem, $\A_0 \subset \A_1 \subset \ldots $ i dla każdego $m \in \N$ zachodzi $\A_m \models \psi$, to $\bigcup_{m \in \N} \A_m \models \psi$.
	Zatem $\F^{\text{alg}} \models \psi_{n,k}$. % indeks p?
	Z zupełności $\ACF_p$, gdzie $p$ jest pierwsza, mamy $\ACF_p \models \psi_{n,k}$.
	Ze zwartości i zupełności $\ACF_0$ również $\ACF_0 \models \psi_{n,k}$.

\end{proof}

\begin{df}
	Jeśli $\K \models \ACF$ oraz $I \subseteq K[x_1, x_2, \ldots, x_n ]$ jest ideałem, to wtedy zbiory postaci $$V(I) = \{(a_1, a_2, \ldots, a_n) \in K^n \colon p( a_1, a_2, \ldots, a_n ) = 0 \text{ dla wszystkich } p \in I\}$$ nazywamy \textit{zbiorem algebraicznym}.
\end{df}


\begin{obs}
	Ponieważ $K[x_1, x_2, \ldots, x_n ]$ jest noetherowski, to ideał $I$ jest generowany przez skończenie wiele wielomianów $p_1, p_2, \ldots, p_k$.
	Czyli zbiór $V(I)$ jest definiowalny jako:
	$$V(I) = \{ \bar{a} \in K^n \colon p_1(\bar{a}) = 0 \wedge p_2(\bar{a}) = 0 \wedge p_n(\bar{a})=0 \}.$$
	%CHECK
\end{obs}

\begin{df}
	 Zbiór konstruowalny to kombinacja borelowska zbiorów konstruowalnych.
\end{df}

\begin{tw}[Chevalley'a]
	 Jeśli $\K \models \ACF$, $V \subseteq K^n$ to zbiór konstruowalny, $f \colon K^n \to K^l$ przekształcenie dane wielomianami $q_1, q_2, \ldots, q_l$,
	 wówczas $f(V)$ jest konstruowalny.
	 W szczególności rzut zbioru konstruowalnego jest konstruowalny.
\end{tw}
\begin{proof}
	\[
		f(V) = \{ (y_1, y_2, \ldots, y_l ) \colon \exists x_1, x_2, \ldots, x_n \in V (q_1(\bar{x}) = y_1, \ldots q_l(\bar{x}) = y_l)  \}.
	\]
	Zatem $ f(V) $ jest definiowalny formułą bezkwantyfikatorową, a zatem konstruowalny.
\end{proof}

\begin{tw}[słaba wersja twierdzenia Hilberta o zerach]
	Niech $  \K \models \ACF $ i niech $  I \subseteq K[x_1, x_2, \ldots, x_n ] $ będzie ideałem własnym, tj. $  1 \in I $.
	Wówczas $  V(I) \neq \emptyset $.
\end{tw}
\begin{proof}
	 Niech ideał $I$ będzie generowany przez $  p_1, p_2, \ldots, p_k  $.
	 Niech $  M \supseteq I$ będzie ideałem maksymalnym.
	 Wtedy $L :=  \sfrac{K[x_1, x_2, \ldots, x_n ]}{M} $ jest ciałem.
	 Zauważmy, że $  L \models \exists x_1, x_2, \ldots, x_n (p_1(\bar{x}) = 0 \wedge \ldots \wedge p_k(\bar{x}) = 0) $.
	 Niech $ L^{\text{alg}} $ będzie algebraicznym domknięciem $ L $.
	Mamy $  L^{\text{alg}} \models \exists x_1, x_2, \ldots, x_n (p_1(\bar{x}) = 0 \wedge \ldots \wedge p_k(\bar{x}) = 0) $.
	Ale $  K \subseteq L^{\text{alg}} $, a $ K , L \models \ACF $, czyli $ K \prec L^{\text{alg}} $.
	Stąd $  K \models \exists x_1, x_2, \ldots, x_n (p_1(\bar{x}) = 0 \wedge \ldots \wedge p_k(\bar{x}) = 0) $
\end{proof}

% WYKŁAD 3, 14 III 2017

\begin{stw}
	 $ \R $ nie ma eliminacji kwantyfikatorów w języku $ \{+, \cdot, -, 0, 1\} $. %TODO "-" jest jednoargumentowy
\end{stw}
\begin{proof}
	Gdyby miało, to $ (\R, +, \cdot, -, 0, 1) $ byłoby silnie minimalne.
	Ale $\exists y (y^2 = x) $ definiuje nieskończony, konieskończony podzbiór $ \R $.
\end{proof}

\begin{df}
	 $ \RCF $ to teoria w języku $ \{ +, \cdot, -, 0, 1, \leq \} $ z następującą aksjomatyką:
	 \begin{enumerate}
		 \item $ \{+, \cdot, -, 0, 1\} $ spełnia aksjomaty ciała,
		 \item po dodaniu $\leq $ mamy ciało uporządkowane, tzn. spełnione jest:
			 \begin{itemize}
				 \item $ \leq $ jest liniowym porządkiem,
				 \item $ \forall x y (x, y \geq 0 \Rightarrow x\cdot y \geq 0) $,
				 \item $\forall x y z (x \leq y \Rightarrow x + z \leq y + z )  $,
			 \end{itemize}

		 \item $ \forall x (x \geq 0 \Rightarrow \exists y ( y^2 = x )) $,
		 \item dla wszystkich nieparzystych $ n \in \N$ zachodzi
			 \[
				 \forall y_0 y_1 \ldots y_n (y_n \neq 0 \Rightarrow \exists x (y_n \cdot x^n + \ldots + y_1 \cdot x + y_0 = 0))
		         \]
	 \end{enumerate}

\end{df}

Motywacją do badania $ \RCF $ jest m. in. 17. problem Hilberta, który brzmi następująco:
\textit{Czy każda funkcja wymierna $ f \in \R(x_1, x_2, \ldots, x_n ) $ taka, że $ f \geq 0 $ jest sumą kwadratów funkcji wymiernych?}

Naszym celem na teraz będzie pokazanie w kilku krokach, że $ \RCF $ ma eliminację kwantyfikatorów.

\begin{tw}
\label{tw:rcf}
	 Niech $ (K, \leq) $ będzie ciałem uporządkowanym.
	 Następujące warunki są równoważne:
	 \begin{enumerate}
		 \item $(K, \leq) \models \RCF $,
		 \item $K(i) := \sfrac{K[x]}{(x^2 + 1)} $ jest acf (ciałem
			 algebraicznie domkniętym),
		 \item $ (K, \leq) $ nie ma żadnego właściwego rozszerzenia
			 algebraicznego uporządkowanego zgodnie z $ \leq $.
	 \end{enumerate}
% TODO w notatkach najpierw było R, potem K. Czy notacja OK?
\end{tw}
% TODO na wykładzie dowód był pokazan później -- czy tu też tak chcemy?

% TODO Jakoś tu nie widzę tw. Sturma...
\begin{wn}
	Jeśli $ (R, \leq) \models \RCF $, to $ (R, \leq) $ ma własność Darboux dla wielomianów:
	jeśli $ p(a) \cdot p(b) < 0 $ oraz $ a <b $, to istnieje $c \in (a, b) $ takie, że $ p(c) = 0 $, gdzie $ p \in R[x] $.
\end{wn}
\begin{proof}
	 Bez utraty ogólności, $ p $ jest nierozkładalny.
	 A zatem, na mocy punktu 2. z twierdzenia powyżej, $ p $ jest stopnia $ 1 $, lub $ 2 $ z $ \Delta < 0$.
	 \begin{itemize}
		 \item Jeśli stopnia $ 1 $, to jest ok.
		 \item Jeśli stopnia $2$ z $\Delta < 0 $, to $ p $ ma stały znak na $ R $.
	 \end{itemize}

\end{proof}

\begin{tw}
	 $ \RCF$ ma eliminację kwantyfikatorów.
\end{tw}
\begin{proof}
	Pokażemy, że zachodzą obydwa warunki dostateczne posiadania q.e. z twierdzenia \ref{tw:qe}.
	\begin{lem} % po co lemat?
		 Niech $ (K, \leq) \subseteq (R, \leq)$ będą rcf-ami,
	        $ \bar{a} \in K, \psi(\bar{x}, y)$ niech będzie formułą bezkwantyfikatorową,
		oraz niech $ b \in R$ będzie takie, że $(R, \leq) \models  \psi(\bar{a}, b)$.
		Bez utraty ogólności $ \psi(\bar{a}, y)$ ma postać:
		\[
			p(y) = 0 \wedge  \bigwedge_{ i = 1 }^n q_i > 0,
		\]
		gdzie $ p, q_i \in K[x], q_i \not \equiv 0$.
	\end{lem}
	Są dwie możliwości:
	\begin{enumerate}
		\item $ p \not \equiv 0$.
			Wtedy $ b$ jest algebraiczne nad $ K$.
			Z charakterystyki rcf-ów (twierdzenie \ref{tw:rcf}), $ b \in K$.
		\item $ p \equiv 0$.
			Wtedy $ \psi$ ma postać $ \bigwedge_{ i } q_i(y) > 0$.
			\\TODO OBRAZEK\\
			Zauważmy, że
			\begin{itemize}
				\item $ c_i < b$ i $ c_i \in K \cup \{-\infty\}$,
				\item $ d_i > b$ i $ d_i \in K \cup \{+\infty\}$.
			\end{itemize}
			Niech $ c := \max (c_1, \ldots, c_n)$ oraz $ d := \min (d_1, \ldots, d_n)$.
			Wtedy dla dowolnego elementu $ b' \in (c,d) \cup K $ zachodzi $ (K, \leq) \models \psi(\bar{a},\tilde{b})$.
	\end{enumerate}
	\begin{lem}
		 $\RCF_{\forall}$ to teoria pierścieni uporządkowanych.
	\end{lem}
	\begin{proof}
		 \textbf{($ \subseteq$)} Każda podstruktura ciała uporządkowanego jest pierścieniem uporządkowanym.
		 \\\textbf{($ \supseteq$)} Niech $ (P, \leq)$ będzie pierścieniem uporządkowanym, a $Q$ niech będzie ciałem ułamków $P$.
		 Rozszerzamy $\leq$ w kanoniczny sposób na $Q$ dostając ciało uporządkowane $ (Q, \leq)$.
		 Weźmu maksymalne ciało uporządkowane $(R, \leq)$ będące algebraicznym rozszerzeniem $ (Q, \leq)$
		 (na mocy lematu Kuratowskiego-Zorna).
		 Na mocy charakteryzacji rcf-ów z twierdzenia \ref{tw:rcf} mamy $(R, \leq) \models \RCF$,
		 a zatem $(P, \leq) \models \RCF_{\forall}$.
	\end{proof}
	\begin{df}
		 Niech $(Q, \leq)$ będzie ciałem uporządkowanym.
		 Dowolny rcf będący algebraicznym rozszerzeniem $(Q, \leq)$ uporządkowanym zgodnie z $ \leq$ nazywamy
		 \textit{rzeczywistym domknięciem} $ (Q, \leq)$.
	\end{df}
	\begin{lem}
		 Jeśli mamy pierścień uporządkowany $ (D, \leq)$,
		 $(Q, \leq)$ jest ciałem ułamków $D$ uporządkowanym przez kanoniczne rozszerzenie $\leq$,
		 $ (R, \leq)$ jest rzeczywistym domknięciem $ (Q, \leq)$
		 oraz $ (Q, \leq) \subseteq (K, \tilde{\leq}) \models \RCF$,
		 to istnieje zanurzenie $ (R, \leq)$ w $ (K, \tilde{\leq})$ będące identycznością na $ Q$
	\end{lem}
	\begin{proof}
		 TODO
		W tym celu korzystamy z twierdzenia Sturma:
		\begin{tw} \label{tw:sturma}
			Niech $(R, \leq)$ to rcf, $p \in R[X]$ niech nie ma
			pierwiastków wielokrotnych, niech $a, b \in R$ będą
			takie, że $p(a) \neq 0 \neq p(b)$ oraz $a < b$.
			Rozważmy ciąg
			\begin{align*}
				&p_0 = p \\
				&p_1 = p' \\
				&\vdots \\
				&p_{i+2} = -p_i + g \cdot p_{i + 1},
				\text{ gdzie } g \text{ jest dobrane tak, by }
				\deg p_{i+1} > \deg  p_{i+2} \\
				&\vdots \\
				&p_s = \text{NWD}(p, p')
				\text{ (stała niezerowa w } R\text{)}
			\end{align*}
			Niech $v(p, x)$ oznacza liczbę zmian znaków w ciągu
			$$p_0(x), p_1(x), \ldots, p_s(x),$$
			(zera ignorujemy). Wtedy liczba pierwiastków $p$
			w $R \cap (a, b)$ jest równa $$v(p, a) - v(p, b).$$
		\end{tw}
		 TODO
	\end{proof}




\end{proof}

\begin{wn}
	$ \RCF$ jest zupełna, rozstrzygalna i \textit{o}-minimalna.
\end{wn}

\begin{uw}
	 Z rozstrzygalności $ \RCF$ wynika rozstrzygalność ,,geometrii elementarnej".
\end{uw}

\begin{df}
	 Niech $ (R, \leq)$ będzie rcf-em.
	 Zbiór \textit{semialgebraiczny} to $ S \subseteq R^n$ będący skończoną sumą zbiorów postaci
	 \[
		 \{\bar{x} \in R^n \colon p_1(\bar{x}) = \ldots = p_l(\bar{x}) = 0, q_1(\bar{x}) > 0, \ldots, q_k(\bar{x}) > 0 \}
	 \]
\end{df}

\begin{uw}
	 Rodzina zbiorów semialgebraicznych jest domknięta na rzutowania.
\end{uw}

% WYKŁAD 4, 21 III 2017

Zastanówmy się nad następującą rzeczą: jakie ciała są porządkowalne?
Wiemy bowiem, że $ \C$ lub ciała charakterystyki $ p$ nie są porządkowalne.

\begin{df}
	 $ K$ jest ciałem \textit{formalnie rzeczywistym}, jeśli $ -1$ nie jest sumą kwadratów, tzn. $ -1 \not \in \Sigma K^2$.
\end{df}
\begin{df}
         Niech $ K$ będzie ciałem formalnie rzeczywistym.
         \textit{Stożkiem} w $ K$ nazywamy taki zbiór $P$,
	 że $\Sigma K^2 \subseteq P \subseteq K$, $P$ jest domknięty na działania $+, \cdot$ oraz $-1 \not \in P$.
\end{df}

\begin{uw}
	W każdym ciele formalnie rzeczywistym $K$ jest przynajmniej jeden stożek, mianowicie $\Sigma K^2$.
\end{uw}
\begin{stw}
	Jeśli $P$ jest stożkiem oraz $-a \not \in P$, to zbiór $$P[a] := \{x + a \cdot y \colon x, y \in P\}$$ też jest stożkiem.
\end{stw}
\begin{proof}
	Wystarczy pokazać, że $-1 \not \in P[a]$.
	Załóżmy przeciwnie: niech $-1 = x + a \cdot y$ dla pewnych $x, y \in P$.
	Musi być $y \neq 0$, bo $-1 \not \in P$.
	Więc
	\begin{align*}
		-a \cdot y &= x + 1 \\
		-a &= \frac{1}{y} \cdot (x + 1) \\
		-a &= \frac{1}{y^2} \cdot y \cdot (x + 1).
	\end{align*}
	Prawa strona ostatniego równania należy do  $P$, tak więc sprzeczność.
\end{proof}
\begin{wn}
	 Niech $K$ będzie ciałem formalnie rzeczywistym a $P$ stożkiem w $K$.
	 $P$ rozszerza się do stożka $Q$ takiego, że dla każdego $a \in K$ jest $a \in Q$ lub $-a \in Q$.
\end{wn}
\begin{proof}
	 Niech $Q$ będzie maksymalnym stożkiem zawierającym $P$.
	 Weźmy $a \in K$ i załóżmy, że $a \not \in P$.
	 Wtedy $Q[-a]$ jest stożkiem, a z maksymalności $Q$ mamy $-a \in Q$.
\end{proof}

\begin{wn}
	\label{wn:forrz}
	 Niech $K$ będzie ciałem formalnie rzeczywistym oraz niech $a \not \in \Sigma K^2$.
	 Wówczas $K$ ma uporządkowanie $\leq$, w którym $a < 0$.
\end{wn}
\begin{proof}
	 $\Sigma K^2$ jest stożkiem.
	 Wystarczy więc w dowodzie wniosku powyżej za $P$ wziąć $\Sigma K^2$ i otrzymany maksymalny stożek $Q$ porządkuje $K$.
\end{proof}

\begin{tw}
	 17. problem Hilberta ma pozytywne rozwiązanie:
	 jeśli $f \in \R(x_1, x_2, \ldots, x_n)$ oraz $\forall \bar{x} \in \R^n \, f(\bar{x}) \geq 0$,
	 to są $g_1, g_2, \ldots \in \R(x_1, x_2, \ldots, x_n)$ takie, że $f = \Sigma_i g_i^2$
\end{tw}
\begin{proof}
	 Załóżmy przeciwnie.
	 $K := \R(x_1, x_2, \ldots, x_n) $ jest ciałem formalnie rzeczywistym.
	 Założyliśmy, że $f \not \in \Sigma K^2$, dlatego na mocy wniosku \ref{wn:forrz} istnieje takie uporządkowanie $\leq$,
	 że $f < 0$.
	 Zauważmy, że $\leq$ rozszerza zwykły porządek na $\R$ (bo każdy nieujemny element $\R$ w $\leq$ jest kwadratem).
	 Niech $(L, \leq)$ będzie rzeczywistym domknięciem $(K, \leq)$.
	 \[
	 	 (L, \leq) \models f(x_1, x_2, \ldots, x_n ) < 0
	 \]
	 \[
	 	 (L, \leq) \models \exists \bar{x}  \, f(\bar{x}) < 0.
	 \]
	 Z modelowej zupełności $\RCF$ mamy $(\R, \leq) \preccurlyeq (L, \leq)$, więc
	 \[
	 	 (\R, \leq) \models  \exists \bar{x}  \, f(\bar{x}) < 0,
	 \]
Zatem sprzeczność.
\end{proof}

% Tu idą uzupełnienia luk w dowodzie tego i owego. Trzeba chyba to uporządkować.

Teraz pora uzupełnić luki powstałe w toku dochodzenia do eliminacji
kwantyfikatorów dla $\RCF$. Najpierw udowodnimy Twierdzenie \ref{tw:rcf}.

\begin{proof}[Dowód Twierdzenia \ref{tw:rcf}]

	\textbf{($2 \Rightarrow 1$)}
	Jasne -- przy założeniu (2) $K(i)$ jest jedynym właściwym rozszerzeniem
	$K$.

	\textbf{($1 \Rightarrow 3$)}
	Załóżmy (1). Niech $K \ni a \geq 0$ oraz niech
	$\sqrt{a} \notin K$. Weźmy $K(\sqrt{a})$. Rozważmy $P \subseteq
	K(\sqrt{a})$ zbiór elementów postaci
	\[
		\sum_{i=1}^{n} b_i (c_i + d_i \sqrt{a})^2,
	\]
	gdzie $b_i, c_i, d_i \in K, b_i \geq 0$. Pokażemy, że $P$ jest stożkiem
	w $K(\sqrt{a})$, co udowodni, że $K(\sqrt{a})$ można uporządkować
	zgodnie z $\leq$ -- wbrew (1).

	Załóżmy, że $-1 = \sum_{i=1}^n b_i (c_i + d_i \sqrt{a})^2$.%TODO check ^2
	Wtedy
	$$ -1 = \sum_{i=1}^n b_i c_i^2 + b_i d_i^2 a \geq 0$$
	równość -- bo $\sqrt{a} \notin K$,
	nierówność -- bo $b_i, c_i^2, d_i^2, a \geq 0$.
	Czyli faktycznie -- wszystkie nieujemne elementy $K$ są kwadratami.
	% TODO sprzeczność -- błyskawica

	Teraz załóżmy, że wielomian $p(x) \in K(x)$ jest nieparzystego stopnia
	bez pierwiastków w $K$. Bez utraty ogólności, niech $P$ będzie
	najmniejszego stopnia i nierozkładalny. % TODO najmniejszego? ->notatki
	Niech $deg p = 2n + 1$. Wiemy, że $\sfrac{K(X)}{(p)}$ nie jest
	formalnie rzeczywiste. W takim razie możemy napisać: % TODO sfrac?
	\[
		- 1 = \sum_{j=1}^m h_i^2(x) + g(x) \cdot p(x)
	\]
	gdzie, bez utraty ogólności, $\sum_{j=1}^m h_i^2(x)$ jest stopnia
	parzystego $\leq 4n$, a $g(x)$ jest stopnia nieparzystego $\leq 2n + 1$.
	Zatem $g$ ma pierwiastek w $K$ -- $g(\alpha) = 0$. Stąd
	\[
		- 1 = \sum_{j=1}^m h_i^2(\alpha) + 0,
	\]
	sprzeczność.

	\textbf{($3 \Rightarrow 2$)}
	Najpierw pokażemy, że każdy wielomian $p(x) \in P[x]$ ma pierwiastek w
	$K(i)$: udowodnimy to przez indukcję po $m$, gdzie
	$$deg(p) = d = 2^m(2n + 1).$$
	$m = 0$ już mamy. Weźmy $m > 0$.
	Dla każdego $k \in \N$ rozważmy wielomian
	\[
		f_k(X, Y_1, \ldots, Y_d) =
		\prod_{1 \leq \mu \leq \lambda \leq d}
		X - Y_\mu - Y_\lambda - k Y_\mu Y_\lambda
	\]
	Zauważmy, że jeśli popatrzymy na $f_k$ jako na wielomian z
	$(K[Y_1, \ldots, Y_n])[X]$, to współczynnik przy każdej potędze $X$
	jest kombinacją wielomianów
	\begin{align*}
		&Y_1 + \ldots + Y_d \\
		&Y_1 Y_2 + Y_1 Y_3 + \ldots + Y_{d-1} Y_d \\
		&Y_1 Y_2 Y_3 + Y_1 Y_2 Y_4 + \ldots + Y_{d-2} Y_{d-1} Y_d
	\end{align*}
	Teraz, jeśli $\alpha_1, \ldots, \alpha_d$ to wszystkie pierwiastki
	$p$, to $f_k(X, \alpha_1, \ldots, \alpha_d)$ % TODO ale że co? uzupełnij
	(z obserwacji nt. $f_k$ oraz z wzorów Viete'a).
	\[
		deg f_k(X, \alpha_1, \ldots, \alpha_d) = {d \choose 2} =
		2^{m-1}(2s + 1)
	\]
	Czyli dla każdego $k$, $f_k(X, \alpha_1, \ldots, \alpha_d)$ ma
	pierwiastek w $K(i)$ -- stąd dla każdego $k$ istnieją
	$1 \leq \mu \leq \lambda \leq d$ takie, że
	$k \alpha_\lambda \alpha_\mu - \alpha_\lambda - \alpha_\mu \in K(i)$.
	Z zasady szufladkowej są takie $\lambda < \mu$, że
	$\alpha_\lambda \alpha_\mu \in K(i)$. Zatem
	$\alpha_\lambda + \alpha_\mu \in K(i)$.
	\begin{equation}
		(X - \alpha_\lambda) (X - \alpha_\mu) =
		X^2 - (\alpha_\lambda + \alpha_\mu) X + \alpha_\lambda \alpha_\mu
	\end{equation}
	Z ogólnych rozważań nad $\Delta$, ten wielomian ma dwa pierwiastki
	w $K(i)$. W takim razie $\alpha_\lambda, \alpha_\mu \in K(i)$.
	Czyli $p(x)$ ma pierwiatek w $K(i)$. Jeśli teraz $p \in K(i)[X]$ to
	rozważmy $\bar{p}$, tj. wielomian otrzymany z $p$ poprzez sprzężenie
	współczynników. $p \bar{p} \in K[X]$, więc ma pierwiatek $\alpha$
	w $K(i)$. $p(\alpha) = 0$, lub $\bar{p}(\alpha) = 0$ -- w którym to
	przypadku $p(\bar{\alpha}) = 0$.
\end{proof}

\begin{proof}[Dowód Twierdzenia Sturma, \ref{tw:sturma}] % TODO ref ?
	Zauważmy, że nie ma $\alpha \in (a, b)$ takiego, że
	$p_i(\alpha) = p_{i+1}(\alpha) = 0$. Pozostaje rozważyć takie $\alpha$,
	że dla pewnego $i$, $p_i(\alpha) = 0$, i zobaczyć, co się dzieje ze
	znakami $p_{i-1}(\alpha)$, $p_{i}(\alpha)$, $p_{i+1}(\alpha)$ na lewo
	i na prawo od $\alpha$.


	TODO skończ.
\end{proof}

\section{Typy}

\begin{df}
	\textit{Typ} (czy też $n$-\textit{typ}) nad teorią $T$ to maksymalny, niesprzeczny z $T$ zbiór formuł o zmiennych wolnych należących do zbioru $\{x_1, x_2, \ldots, x_n\}$.
		Jeśli $\A$ to struktura oraz $D \subseteq A$, to \textit{typem nad $\A$ z parametrami z $D$} nazywamy typ nad teorią $\Th(\A_D)$, gdzie $\A_D$ to rozszerzenie struktury $\A$ o stałe dla każdego elementu z $D$.
		Podzbiór typu nazywamy \textit{typem częściowym}.
\end{df}

\begin{ozn}
	Typy typowo oznaczamy przez $p(\bar{x}), q(\bar{x})$, natomiast:
	 ~\begin{itemize}
		 \item przez $S_n(T)$ oznaczamy zbiór $n$-typów nad $T$,
		 \item przez $S_n^{\A}(D)$ oznaczamy zbiór $n$-typów nad $\A$ z parametrami z $D \subseteq A$.
	 \end{itemize}
\end{ozn}

\begin{df}
	Typ $p(\bar{x})$ jest \textit{realizowany} w $\A$, jeśli jest takie $\bar{a} \in A^n$, że dla każdej formuły $\phi(\bar{x}) \in p(\bar{x})$ zachodzi $\A \models \phi(\bar{a})$.
	W przeciwnym przypadku mówimy, że $\A$ \textit{omija} typ $p(\bar{x})$.
\end{df}

\begin{prz}
	Rozważmy następujące typy nad $(\Q, \leq)$ z parametrami z $\N$:
	\begin{itemize}
		\item $p_1(x) = \{\phi(x, \bar{n}) \colon \Q \models \phi(\frac{1}{2}, \bar{n})\}$,
		\item $p_2(x)$ będący maksymalnym, niesprzecznym rozszerzeniem zbioru formuł $\{x > n \colon n \in \N\}$.
	\end{itemize}
	Typ $p_1$ jest realizowany, natomiast $p_2$ jest ominięty.
\end{prz}

\begin{stw}
	Jeśli $p(\bar{x})$ jest typem nad $\A$ z parametrami z $D \subseteq A$,
	 to istnieje $\tilde{\A} \succcurlyeq \A$ realizujący $p$.
\end{stw}
\begin{proof}
	Teoria $\DiagEl(\A) \cup p(\bar{x})$ jest niesprzeczna, a jej model jest elementarnym rozszerzeniem $\A$.
\end{proof}

% WYKŁAD 5, 28 III 2017

Pytanie: jakie typy można omijać?


\begin{df}
	Typ $p(\bar{x})$ nad teorią $T$ (być może częściowy) jest \textit{izolowany},
	jeśli istnieje formuła $\phi(\bar{x})$ taka, że $T + \exists \bar{x}\, \phi(\bar{x})$ jest niesprzeczne oraz dla każdego $\psi(\bar{x}) \in p(\bar{x})$, $T \vdash \forall \bar{x} \, (\phi(\bar{x}) \Rightarrow \psi(\bar{x}))$
\end{df}

\begin{stw}
	Jeśli typ $p(\bar{x})$ nad $T$ jest izolowany, to jest realizowany w każdym modelu $T + \exists \bar{x} \, \phi(\bar{x})$.
	W szczególności, jeśli $T$ jest zupełna, to $p(\bar{x})$ jest realizowany w każdym modelu $T$.
\end{stw}

\begin{tw}[O omijaniu typów]
	\label{tw:omi}
	Niech $T$ będzie teorią w przeliczalnym języku, a $p(\bar{x})$ niech będzie typem nad $T$, który nie jest izolowany.
	Wówczas istnieje model $\A \models T$, który omija $p$.
\end{tw}
\begin{proof}
	Do języka dodajemy nowe stałe $c_1, c_2, \ldots$
	Niech $\phi_1, \phi_2, \ldots $ będzie listą wszystkich zdań w tym rozszerzonym języku, a $\psi_1(x), \psi_2(x), \ldots$ będzie listą wszystkich formuł z jedną zmienną wolną w rozszerzonym języku.
	Definiujemy łańcuch teorii $T = T_0 \subseteq T_1 \subseteq T_2 \subseteq \ldots$
	w sposób następujący:
	\begin{itemize}
		\item \textit{Krok $3i+1$:}
			\[
				 T_{3i+1} := \left\{\begin{array}{lr}
						 T_{3i} + \phi_i, & \text{jeśli niesprzeczna,}\\
						 T_{3i} + \neg \phi_i, & \text{w p.p.}
				 \end{array}\right\}
			\]

		\item \textit{Krok $3i+2$:}
			Niech $c_j$ będzie pierwszą stałą nie występującą w teorii $T_{3i+1}$.
			Wówczas $T_{3i+2} := T_{3i+1} + \exists x \, \psi_i(x) \Rightarrow \psi_i(c_j)$.
		\item \textit{Krok $3i+3$:}
			Niech $\bar{c}_i$ będzie $i$-tą $n$-tką nowych stałych.
			$T_{3i+2}$ jest postaci $T + \delta(\bar{c}_i, \bar{d})$, gdzie $\delta(\bar{x}, \bar{y})$ jest formułą a $\bar{d}$ to nowe stałe spoza $\bar{c}_i$.
			Wiadomo, że $T + \exists \bar{x} \, \exists \bar{y} \, \delta(\bar{x}, \bar{y})$ jest niesprzeczna, a zatem z nieizolowalności $p(\bar{x})$ istnieje $\pi(\bar{x}) \in p(\bar{x})$ takie, że $T + \exists \bar{x} \, \exists \bar{y} \, \delta(\bar{x}, \bar{y}) \not \vdash \pi(\bar{x})$.

			Bierzemy $T_{3i+3} := T_{3i+2} + \neg \pi(\bar{c}_i)$.
	\end{itemize}
	Zauważmy, że $\bigcup_n T_n$ jest zupełna i ma własność zaświadczania przez stałe.
	Niech $\B \models \bigcup_n T_n$ oraz niech $\widetilde{\A} \preccurlyeq \B$, gdzie uniwersum $\widetilde{\A}$ to $\{c_1^{\B}, c_2^{\B}, \ldots\}$.
	$\widetilde{\A}$ jest podstrukturą $\B$ na mocy zaświadczania przez stałe, a podstrukturą elementarą na mocy kryterium Vaughta -- wszystko z powodu kroków $3i+2$.
	Niech $\A$ będzie reduktem $\widetilde{\A}$ do języka $T$.
	$\A \models T$, bo $\widetilde{\A} \models \bigcup_n T_n \supseteq T$.
	Ponadto $\A$ omija typ $p$ na mocy kroków $3i+3$.
	% TODO wyjaśnij to z LAK

\end{proof}
\begin{uw}
	 W ten sam sposób można naraz ominąć przeliczalnie wiele typów.
\end{uw}

\begin{tw}
	 Niech $T$ będzie zupełną teorią w przeliczalnym języku.
	 Następujące warunki są równoważne:
	 \begin{enumerate}
		 \item $T$ jest $\omega$-kategoryczna,
		 \item Dla każdego $n \in \N$, każdy typ w $S^n(T)$ jest izolowany,
		 \item Dla każdego $n \in \N$, $|S^n(T)| < \aleph_0$,
		 \item Dla każdego $n \in \N$, z dokładnością do równoważności w $T$ istnieje tylko skończenie wiele formuł ze zmiennymi wolnymi $x_1, x_2, \ldots, x_n$.
	 \end{enumerate}
\end{tw}
\begin{proof}
	\textbf{($4 \Rightarrow 3$)} Jeśli dla danego $n$, $\phi_1(\bar{x}), \phi_2(\bar{x}), \ldots, \phi_k(\bar{x})$  to wszystkie formuły z punktu 4, to typ z $S^n(T)$ możemy określić na co najwyżej $2^k$ sposobów -- wskazując, które formuły do niego należą.

	\textbf{($3 \Rightarrow 2$)} Niech $p_1, p_2, \ldots, p_k$ to wszystkie typy w $S^n(T)$.
	Weźmy typ $p_i$.
	Dla każdego $j \neq i$ istnieje formuła $\phi_{ij}(\bar{x})$ taka, że jednocześnie $\phi_{ij}(\bar{x}) \in p_i$ oraz $\neg \phi_{ij}(\bar{x}) \in p_j$.
	Wtedy $p_i$ jest izolowany przez $\bigwedge_{i \neq j} \phi_{ij}(\bar{x})$.

	\textbf{($2 \Rightarrow 1$)} Niech $\A, \B$ będą modelami $T$.
	Pokażemy, że Duplikator wygrywa $G^{\infty}(\A, \B)$, co da 1.
	Strategia Duplikatora:
	jeśli po $k$ rundach wybrane są elementy $a_1, a_2, \ldots, a_k \in \A$ oraz $b_1, b_2, \ldots, b_k \in \B$, to $\text{tp}^{\A}(a_1, a_2, \ldots, a_k) = \text{tp}^{\B}(b_1, b_2, \ldots, b_k)$.
	Tak można zrobić, bo np. $\text{tp}^{\A}(a_1, a_2, \ldots, a_{k+1})$ izolowany formułą
	$\phi(x_1, x_2, \ldots, x_{k+1})$, ale $\exists x_{k+1} \, \phi(x_1, x_2, \ldots, x_{k+1}) \in \text{tp}^{\A}(a_1, a_2, \ldots, a_k)$.

	\textbf{($1 \Rightarrow 2$)} Załóżmy nie wprost, że w $S^n(T)$ jest typ nieizolowany $p(\bar{x})$.
	Wtedy istnieje przeliczalny model $\A$ realizujący $p$ oraz drugi model przeliczalny $\B$ omijający $p$.
	Oczywiście $\A \not \simeq \B$.

	\textbf{($2 \Rightarrow 3$)}
	Załóżmy nie wprost, że dla pewnego $n$ typów w $S^n(T)$ jest nieskończenie wiele.
	Zauważmy, że istnieje formuła $\psi(\bar{x})$ taka, że zarówno ona, jak i jej negacja występują w nieskończenie wielu typach.
	W przeciwnym przypadku zbiór $\{\phi(\bar{x}) \colon \phi \text{ występuje w nieskończenie wielu typach}\}$ byłby typem nieizolowanym.
	Podobnie, jeśli $\psi_1(\bar{x})$ występuje w nieskończenie wielu typach, to istnieje $\psi_2(\bar{x})$ takie, że zarówno $\psi_1(\bar{x})\wedge \psi_2(\bar{x})$, jak i $\psi_1(\bar{x})\wedge \neg \psi_2(\bar{x})$ występują w nieskończenie wielu typach.
	Istnieje typ $\{\psi_i(\bar{x}) \colon i \in \N\}$ taki, że dla każdego $k $ koniunkcja $\psi_1(\bar{x}) \wedge \psi_2(\bar{x}) \wedge \ldots \wedge \psi_k(\bar{x})$ występuje w nieskończenie wielu typach.
	Ten typ nie może być izolowany.

	\textbf{($3 \Rightarrow 4$)}
	Formuły nierozróżnialne przez typy są równoważne.
\end{proof}


%\section{Modele małe i duże}
\begin{df}
	 Niech $\kappa$ będzie liczbą kardynalną.
	 Model $\A$ nazywamy $\kappa$-nasyconym, jeśli dla każdego $D \subseteq A$, $|D| < \kappa$, każdy typ w $S^{\A}_n(D)$ jest realizowany w $\A$.
	Model $\A$ jest nasycony, jeśli jest $|A|$-nasycony.
\end{df}

\begin{uw}
	Nasycenie wystarczy sprawdzać dla $1$-typów (przy założeniu $\kappa \geq \aleph_0$).
\end{uw}

\begin{stw}
	 Jeśli $T$ jest przeliczalną i zupełną teorią, to każde dwa przeliczalne i nasycone modele $T$ są izomorficzne.
\end{stw}
\begin{proof}
	 TODO
\end{proof}
%TODO przy typach jednolicić indeks górny/dolny
\begin{stw}
	Zupełna teoria $T$ ma przeliczalny model nasycony wtedy i tylko wtedy, gdy $\forall n \,|S_n(T)| \leq \aleph_0$.
\end{stw}
\begin{proof}
	 \textbf{($\Rightarrow$)} Oczywiste.
	 \\\textbf{($\Leftarrow$)} Jeśli w $S_n(T)$ jest tylko przeliczalnie wiele typów dla dowolnego $n$, to również dla dowolnej skończonej krotki parametrów $a_1, a_2, \ldots, a_k \in A$, gdzie $\A \models T$, jest tylko przeliczalnie wiele typów o $n$ zmiennych z parametrami $a_1, a_2, \ldots, a_k$ (bo to jeden z możliwych typów nad $k+n$ zmiennymi).
	 Konstruujemy model przeliczalnie nasycony jako sumę łańcucha elementarnego $\A_1, \A_2, \ldots$ złożonego z przeliczanych modeli $T$ -- po kolei realizując wszystkie typy ze wszystkich krotek parametrów pojawiających się w trakcie konstrukcji.

\end{proof}

\begin{df}
	$\A$ jest \textit{modelem atomowym}, jeśli dla dowolnego $n \in \N$ oraz dowolnych $a_1, a_2, \ldots, a_n \in A$ typ $\text{tp}^{\A}(a_1, a_2, \ldots, a_n)$ jest izolowany.
\end{df}
%TODO wyjaśnij oznaczenie tp^{\A}(...)
\begin{df}
	Model $\A \models T$ jest \textit{modelem pierwszym} dla $T$,
	jeśli dla dowolnego modelu $\B \models T$ zachodzi $\A \preccurlyeq \B$.
\end{df}
\begin{stw}
	 Model $\A \models T$ jest pierwszy wtedy i tylko wtedy, gdy jest przeliczany i atomowy.
\end{stw}
\begin{proof}
	\textbf{($\Rightarrow$)} Oczywiste (ze względu na twierdzenie \ref{tw:omi} o omijaniu typów).
	\\\textbf{($\Leftarrow$)}
	Załóżmy, że $\A$ jest przeliczalny i atomowy, i niech $A = \{a_1, a_2, \ldots, a_n\}$.
	Weźmy $B \models T$.
	\\TODO
\end{proof}
%TODO ta gdzie n \in \N czasem powinno być \N^+
\begin{wn}
	 Każde dwa modele pierwsze dla danej teorii są izomorficzne.
\end{wn}
\begin{proof}
	Obydwa są przeliczalne i atomowe, więc istnieje dla nich strategia w $G^{\infty}$.
\end{proof}

\begin{wn}
	Jeśli dla każdego $n$ jest $|S^n(T)| \leq \aleph_0$, to $T$ ma model pierwszy.
	W szczególności, jeśli $T$ ma model przeliczalny i nasycony, to ma model pierwszy.
\end{wn}
\begin{proof}
	 Skoro wszystkich typów jest co najwyżej przeliczalnie wiele, to nieizolowanych też.
	 Można je wszystkie ominąć w jednym modelu przeliczalnym.
\end{proof}

Ile modeli przeliczalnych może mieć przeliczalna i zupełna teoria?
\begin{itemize}
	\item $1$, np. $\DLO$,
	\item $2$ mieć nie może (Vought),
	\item $3, 4, 5, \ldots$ -- może mieć,
	\item $\aleph_0$, np. $\Th(\Z, S)$,
	\item $\aleph_1$ przy założeniu $\aleph_1 <  \mathfrak{c}$ -- nie wiadomo, hipoteza Voughta mówi, że nie może tyle mieć,
	\item $ \mathfrak{c}$, np. $\RCF$.
\end{itemize}
Więcej być nie może.


% WYKŁAD 6, 4 IV 2017

\section{Przeliczalna pełność i zwartość dla $\FO(\exists^{\geq \aleph_1})$}

Kwantyfikator $\exists^{\geq \aleph_1}$ czytamy jako ,,istnieje nieprzeliczalnie wiele".

\begin{df}[System dowodowy dla $\FO(\exists^{\geq \aleph_1})$]
	 Dla dowolnego rozsądnego systemu dowodowego dla $\FO$ (w szczególności umożliwiającego zamianę zmiennych związanych)
	 dodajemy następujące aksjomaty i reguły:
	 \\\mbox{  }\textbf{(A1)} $\neg \exists^{\geq \aleph_1} x \, (x = y \vee x = z)$,
	 \\\mbox{  }\textbf{(A2)} $\forall x \,(\phi \Rightarrow \psi) \Rightarrow (\exists^{\geq \aleph_1} x \, \phi \Rightarrow \exists^{\geq \aleph_1} x \, \psi)$,
	 \\\mbox{  }\textbf{(A3)} $\exists^{\geq \aleph_1} y \exists x \, \phi \Rightarrow (\exists x \exists^{\geq \aleph_1} y \, \phi \vee \exists^{\geq \aleph_1} \exists y \, \phi)$.

\end{df}

\begin{uw}
	Te reguły/aksjomaty byłyby również poprawne dla $\exists^{\geq \aleph_0} x$.
\end{uw}

\begin{ozn}
	Dla krótkości piszemy ,,$\q$" zamiast ,,$\exists^{\geq \aleph_1} x$".
\end{ozn}

\begin{prz}
	Parę formuł dowodliwych w $\FO(\q)$:
	\begin{itemize}
		\item $\q x \, \phi \Rightarrow \exists x \, \phi$,
		\item $\exists x \,\q y \, \phi \Rightarrow \q y \, \exists x \, \phi$,
		\item $\q x \, (\phi \vee \psi) \iff \q x \, \phi \vee \q x \, \psi$,
		\item $\q x (\phi \wedge \psi) \iff \phi \wedge \q x \, \psi$, gdy $x \not \in \mathsf{FV}(\phi)$,
		\item $\q x \, \phi \wedge \neg \q x \, \psi \Rightarrow \q x \, (\phi \wedge \neg \psi)$.
	\end{itemize}
\end{prz}
Udowodnijmy przykładowo pierwszą z formułę z listy:
\begin{proof}
	 Załóżmy, że $\neg \exists x \, \phi$.
	 Wtedy $\forall x \, (\phi \Rightarrow (x = y \vee x = z))$.
	 Załóżmy $\q x \, \phi$.
	 Na mocy (A2) $\q x \, (x = y \vee x = z)$.
	 Jest to sprzeczne z (A1).
\end{proof}

Naszym celem będzie pokazanie następującego twierdzenia:
\begin{tw}[Przeliczalna pełność dla $\FO(\q)$]
	Jeśli $T$ to \underline{przeliczalna} teoria w $\FO(\q)$, a $\phi$ to zdanie takie, że $T \vdash \phi$,
	to istnieje dowód $\phi$ z $T$ w naszym systemie dowodowym.
\end{tw}
Z powyższego płynie następujący
\begin{wn}[przeliczalna zwartość $\FO(\q)$]
	 Jeśli $T \vdash \phi$, gdzie $T$ i $\phi$ są jak wyżej, to istnieje skończone $T_0 \subseteq T$ takie, że $T_0 \vdash \phi$.
\end{wn}

\begin{df}
	Parę $(\A, Q)$, gdzie $\A$ jest modelem dla $\FO$ oraz $Q \subseteq \mathcal{P}(A)$, nazywamy \textit{słabym modelem} dla $\FO(\q)$,
	jeśli nasz system dowodowy jest poprawny przy następującej interpretacji:
	\[
		(\A, Q) \models \q x \, \phi[v] \text{ wtedy i tylko wtedy, gdy } \{a \in A \colon (\A, Q) \models \phi[v[a/x]]\} \in Q.
	\]

\end{df}

\begin{lem}
\label{lem:slaby_model}
	Niech $T$ będzie zupełną niesprzeczną teorią w $\FO(\q)$ mającą własność zaświadczania przez stałe.
	Wtedy $T$ ma słaby model.
\end{lem}
\begin{proof}
	 Podobny, jak dla $\FO$, gdzie
	 \[
		 Q = \{\{ b \in A \colon (\A, Q) \models \phi[b/x]\} \colon \phi \text{ takie, że } T \vdash \q x \, \phi \}.
	 \]
	 (To jest definicja indukcyjna.)
\end{proof}

\begin{lem}
	Nasz system dowodowy $\FO(\q)$ spełnia twierdzenie o omijaniu typów,
	tj. jeśłi $T$ jest przeliczalną teorią a $\{p_n(\bar{x})\}_{n \in \N}$ to przeliczalna rodzina typów częściowych nieizolowanych nad $T$,
	to istnieje (przeliczalny) słaby model $(\A, Q) \models T$ omijający wszystkie typy $p_n$.
\end{lem}
\begin{proof}
	Taki, jak dla $\FO$, przy pomocy lematu \ref{lem:slaby_model}.
\end{proof}

TODO Reszta tego wykładu o $\FO(\mathsf{Q})$.

% WYKŁAD X, 11 IV 2017

\section{Parę słów o nieprzeliczalnej kategoryczności}

W tym rozdziale zakładamy, że teoria $T$ jest przeliczalna i zupełna.

\begin{df}
	Mówimy, że $T$ jest $\kappa$-stabilna, jeśli dla każdego $\A \models T$ i każdego $D \subseteq A$ takiego, że $|D| < \kappa$, dla każdego $n \in \N$ zachodzi $|S_n^{\A}(D)| \leq \kappa$.
\end{df}

\begin{df}
	Parę $(\A, \B)$ nazywamy \textit{parą Voughta dla $T$}, jeśli $\A \preccurlyeq \B \models T$ i istnieje formuła $\psi$ taka, że $\psi^{\A}$ jest nieskończony i $\psi^{\A} = \psi^{\B}$.
\end{df}

\begin{tw}
	 $\kappa$ nieprzeliczalna. Teoria $T$ jest $\kappa$-kategoryczna wtedy i tylko wtedy, gdy $T$ jest $\omega$-stabilna i nie ma par Voughta.
	 W szczególności, jeśli $T$ jest kategoryczna w jakiejś mocy nieprzeliczalnej, to jest kategoryczna we wszystkich.
\end{tw}
\begin{proof}
	 TODO
\end{proof}

{\Large \textbf{II część kursu -- Teorie interpretujące arytmetykę}}

W zamierzeniu: mowa będzie o teoriach mających ,,aksjomatyzować matematykę'' (bądź jakiś jej istotny fragment).
Takimi teoriami są m.in. $\ZFC$ oraz $\PA$.

Aksjomatami $\PA$ są:
\begin{itemize}
	\item aksjomaty nieujemnej części pierścienia dyskretnie uporządowanego, czyli $\PA^-$,
	\item schemat indukcji: dla każdej formuły $\phi(x, \bar{v})$ aksjomat postaci
		\[
			\forall \bar{v} \, ((\phi(0, \bar{v}) \wedge \forall x \, (\phi(x, \bar{v}) \Rightarrow \phi(x+1, \bar{v}))\Rightarrow \forall x \, \phi(x, \bar{v}))
		\]
\end{itemize}

\begin{stw}
	$\ZFC$ interpretuje $\PA$, tzn. istnieją formuły
	\[
		\phi_{\N}(x),\, \phi_+(x, y, z),\, \phi_{\cdot}(x, y, z),\, \phi_{\leq}(x, y),\, \phi_0(x),\, \phi_1(x)
	\]
	takie, że jeśli $\gamma$ jest formułą $\PA$ taką, że $\PA \vdash \gamma$, to $\ZFC \vdash \gamma^{\ZFC}$, gdzie $\gamma^{\ZFC}$ jest formułą $\ZFC$ będącą tłumaczeniem formuły $\gamma$ przy pomocy następujących transformacji:
	\begin{itemize}
		\item $x \leq y \mapsto \phi_{\leq}(x, y)$,
		\item $x + y = z \mapsto \phi_{+}(x, y, z)$
		\item \ldots
		\item $\exists x \, \psi \mapsto \exists x \, (\phi_{\N}(x) \wedge \psi^{\ZFC}) $
		\item $\forall x \, \psi \mapsto \forall x \, (\phi_{\N}(x) \Rightarrow \psi^{\ZFC}) $
	\end{itemize}

\end{stw}
\begin{stw}
	\label{stw:model_pa}
	Modele $\PA$ (również $\PA^-$) mają zawsze postać:
	\[ \underbrace{\big(0, 1, 2,  \ldots\big)}_{\substack{\text{część standardowa,} \\ \text{izomorficzna kopia } \omega}}
\underbrace{\cdots \big(\ldots, a , \ldots\big)\cdots\big(\ldots, \left\lfloor\frac {a + b}{2}\right\rfloor, \ldots\big) \cdots
\big(\ldots, b, \ldots\big)\cdots\big(\ldots, 2\cdot b, \ldots\big)}_{\substack{\text{część niestandardowa,}\\ \text{o typie porządkowym }(\omega^{\ast} + \omega) \cdot \eta}} \]

\end{stw}
\begin{tw}
	 Istnieje formuła $x^y=z$, o której $\PA$ dowodzi wszystkich podstawowych własności wykresu funkcji wykładniczej, tj.
	 \begin{itemize}
		 \item $\forall x \, x^0=1$,
		 \item $\forall x\, y\, z \, (x^y=z \Rightarrow x^{y+1} = x \cdot z)$,
		 \item $\forall x\, y_1\, y_2\, z_1\, z_2 \, (x^{y_1} = z_1 \wedge x^{y_2} = z_2 \Rightarrow x^{y_1 + y_2} = z_1 \cdot z_2)$,
		 \item $\forall x\, y \, \exists z \, (x^y = z)$.
	 \end{itemize}
\end{tw}

\begin{wn}
	 Możemy więc w $\PA$ zdefiniować relację ,,$x$-ty bit w zapisie binarnym $y$ to $1$''.
\end{wn}
\begin{proof}
	Mianowicie tak:
	\[
		 \exists z_1 \, z_2 \, (z_2 \leq 2^x - 1 \wedge y = z_1 \cdot 2^{x+1} + 2^x + z_2)
	\]

\end{proof}
\begin{stw}
	Przy pomocy powyżej zdefiniowanej relacji $\PA$ interpretuje $\ZFC \setminus \{\text{aksjomat nieskończoności}\}$.
\end{stw}
\begin{proof}
	 Formuły $\ZFC$ tłumaczymy tak:
	 \[
		 x \in y \mapsto \text{,,}x\text{-ty bit }y\text{ to  }1\text{''}.
	 \]
	Sprawdzamy, że $\PA$ dowodzi w ten sposób przetłumaczone aksjomaty $\ZFC$.
	Przykładowo:
	\begin{itemize}
		\item aksjomat zbioru pustego: $\exists x \, \forall y \, \neg y \in x$ -- spełniony, bo mamy $0$,
		\item aksjomat wyróżniania: $\forall x \, \exists w \, \forall z \, (z \in w \Leftrightarrow z \in x \wedge \phi(z))$ --
			najpierw sprawdzamy, że $x \in y \Rightarrow x \leq y$.
			Przez indukcję po $u \leq x$ dowodzimy
			\[
				 \psi(u) :\equiv \exists w \, \forall z \leq u \, (z \in w \Leftrightarrow z \in x \wedge \phi(z))
			\]
			\begin{itemize}
				\item $\psi(0)$ -- OK,
				\item $\psi(u) \Rightarrow \psi(u+1)$?
					Weźmy świadka $w_u$ na $\psi(u)$.
					$w_{u+1}$ konstruujemy następująco:
					\begin{itemize}
						\item gdy $u+1 \in x \wedge \phi(u+1)$, to $w_{u+1} := w_u + 2^{u+1}$,
						\item w przeciwnym przypadku $w_{u+1} := w_u$.
					\end{itemize}
					Bierzemy więc $w_x$ będące świadkiem dla $\psi(x)$.
			\end{itemize}


	\end{itemize}

\end{proof}

% WYKŁAD X, 25 IV 2017

\begin{df}
	 Wprowadzamy następującą hierarchię formuł:
\begin{itemize}
\item $\Delta_0$ - klasa formuł, w których wszystkie kwantyfikatory są ograniczone,
\item $\Sigma_n$ - klasa formuł postaci $\exists \bar{v}_1 \forall \bar{v}_2 \ldots \mathsf{Q} \bar{v}_n\, \phi $, gdzie $\phi \in \Delta_0$,
\item $\Pi_n$ - klasa formuł postaci $\forall \bar{v}_1 \exists \bar{v}_2 \ldots \mathsf{Q} \bar{v}_n\, \phi $, gdzie $\phi \in \Delta_0$.
\end{itemize}
\end{df}

\begin{uw}
	 Można zakładać, że $x^y=z$ jest klasy $\Delta_0$.
	 Można też zakładać, że mamy funkcję wykładniczą w języku $\PA$, wraz z jej aksjomatami.
\end{uw}

\begin{stw}
	Relacja $R \subseteq \N^k$ jest \emph{rekurencyjnie przeliczalna} wtedy i tylko wtedy, gdy jest definiowalna w $\N$ formułą klasy $\Sigma_1$.
\end{stw}
\begin{proof}
        \textbf{($\Leftarrow$)} Maszyna może przeszukiwać po kolei każdą $n$-tkę, i jeśli $n$-tka spełniająca formułę istnieje, to zostanie znaleziona w skończonym czasie.
	W przeciwnym wypadku maszyna się nigdy nie zatrzyma.
	\\\textbf{($\Rightarrow$)} Jeśli $M$ jest maszyną świadczącą o rekurencyjne przeliczalności $R$, to $R$ można zdefiniować następującą formułą klasy $\Sigma_1$:


\begin{gather*}
	\exists \langle s_0, s_1, \ldots, s_l  \rangle \, \forall i \leq l \, (s_i \text{ jest pewną konfiguracją maszyny } M \wedge \\
	s_0 = \langle -; x_1, x_2, \ldots, x_k; \text{instrukcja początkowa}  \rangle \wedge\\
s_l = \langle \text{TAK}; \text{coś jeszcze}; \text{instrukcja nieistotna} \rangle \wedge \\
\forall i < l \, s_{i+1} \text{ powstaje z } s_i \text{ zgodnie z definicją  } M )
\end{gather*}
\end{proof}

\begin{wn}
	 Relacja $R$ jest rozstrzygalna wtedy i tylko wtedy, gdy $R$ jest definiowalna w $\N$ zarówno $\Sigma_1$, jak i $\Pi_1$ formułą.
	 (Mówimy wtedy, że $R$ jest $\Delta_1$ formułą.)
\end{wn}
\begin{proof}
	 \textbf{($\Rightarrow$)} Oczywiste.
	 \\\textbf{($\Leftarrow$)} Jeśli $R$ i $\N^k \setminus R$ są rekurencyjnie przeliczalne, to obie są rozstrzygalne.
\end{proof}
\begin{stw}
	 Teoria $\PA^-$ jest $\Sigma_1$-zupełna, tzn. jeśli $\phi$ jest $\Sigma_1$-zdaniem i $\N \models \phi$, to $\PA^- \vdash \phi$.
\end{stw}
\begin{proof}
	Niech $\mathbb{A}$ będzie modelem $\PA^-$.
	Jest on postaci takiej, jak przedstawiono w stwierdzeniu \ref{stw:model_pa}.
	Formuła $\phi$ jest postaci $\exists x \, \delta(x)$, gdzie $\delta$ jest $\Delta_0$-formułą.
	Jeśli $\N \models \exists x \, \delta(x)$, to istnieje $m$ takie, że $\N \models \delta(m)$.
	Ale w takim razie $\A \models \delta(m^{\mathbb{A}})$, i $\A \models \phi$.
	Z dowolności $\mathbb{A}$, $\PA^- \vdash \phi$.
\end{proof} %TODO trochę więcej dopisać; bo to chodzi, że N się zanurza w A i dlatego delta jest też tam spełniona, tak?

\begin{df}
	 Niech
	 \begin{itemize}
		 \item $\underline{0} := 0$,
		 \item $\underline{n} := \underbrace{1+1+ \ldots +1}_{n}$.
	 \end{itemize}
	 Term $\underline{n}$ nazywamy liczebnikiem dla $n$.
\end{df}

\begin{stw}
	 Jeśli $R \subseteq \N^k$ jest rozstrzygalna, to $R$ jest reprezentowalna w $\PA^-$ w następującym sensie:
	 istnieje formuła $\phi(x_1, x_2, \ldots, x_k)$ (np. klasy $\Sigma_1$) taka, że
	 \begin{itemize}
		 \item jeśli $(n_1, \ldots, n_k) \in R$, to $\PA^- \vdash \phi(\underline{n_1}, \ldots, \underline{n_k})$,
		 \item jeśli $(n_1, \ldots, n_k) \not\in R$, to $\PA^- \vdash \neg\phi(\underline{n_1}, \ldots, \underline{n_k})$,
	 \end{itemize}
\end{stw}
\begin{proof}
	Niech $\exists y \, \psi(\bar{x} , y)$ definiuje $R$ oraz $\exists z \, \eta(\bar{x} , z)$ definiuje $\N^k \setminus R$, gdzie $\psi, \eta \in \Delta_0$.
	Jako $\phi(\bar{x})$ bierzemy: $\exists y \, (\psi(\bar{x}, y) \wedge \forall z<y \, \neg \eta(\bar{x} , z))$ i korzystamy z $\Sigma_1$-zupełności $\PA^-$.
\end{proof}
\begin{df}
	 Każdemu wyrażeniu języka $\PA$ możemy przypisać kod będący liczbą naturalną, który nazywamy numerem Gödla.
	 Robimy to np. tak: każdemu symbolowi z języka $\PA$ przyporządkowujemy liczbę naturalną,
	 np. $\neg \mapsto 0$, $\wedge \mapsto 1$, $\exists \mapsto 2$, $0 \mapsto 3$, etc.
	 To przyporządkowanie indukuje przyporządkowanie liczb naturalnych dowolnym ciągom tych symboli (w tym formułom i termom) -- za pomocą struktury teoriomnogościowej wbudowanej w $\PA$ albo też przy pomocy systemu np. szesnastkowego.
	 Liczbę przyporządkowaną termowi/formule $\alpha$ oznaczamy $\gnum{\alpha}$.
\end{df}

\begin{stw}
	 Następujące relacje i funkcje są rozstrzygalne i obliczalne, odpowiednio, a zatem reprezentowalne w $\PA^-$:
	 \begin{itemize}
		 \item $x$ jest numerem Gödla zmiennej,
		 \item $x$ jest numerem Gödla termu,
		 \item $x$ jest numerem Gödla formuły,
		 \item $x$ jest numerem Gödla koniunkcji formuł o numerach Gödla $y$ i $z$,
		 \item $x$ jest numerem Gödla formuły powstałej z formuły o numerze Gödla $y$ przez dopisanie $\exists$,
		 \item	\ldots,
		 \item $x$ jest numerem Gödla aksjomatu $\PA$,
		 \item $x$ jest numerem Gödla dowodu zdania o numerze $y$ w teorii $T$ ($\text{Prov}_T(x, y)$),
		 \item $\emph{\text{num}}(x):= \text{ numer Gödla liczebnika }\underline{x}$,
		 \item $\emph{\text{subst}}(x, y):= \begin{cases}
				 \gnum{\phi(t)}&  \text{ gdy } x = \gnum{\phi(s)},\, y = \gnum{t},\,  t \text{ podstawialny za } s \text{ w } \phi \\
						0 & \text{ w p.p. }
					     \end{cases}$
	 \end{itemize}
\end{stw}

\begin{uw}
	W istocie wszystkie powyższe relacje/wykresy funkcji są definiowalne formułami $\Delta_0(\text{exp})$, tj. formułami klasy $\Delta_0$ z dopuszczeniem termów typu $2^{x^x}$.
\end{uw}

\begin{df}
	$\text{Pr}_T(y) := \exists x \, \text{Prov}(y, x)$.
	(Czyli ta formuła mówi: ,,$x$ jest zdaniem dowodliwym w $T$''.)
\end{df}
\begin{uw}
	 Powyższa własność jest rekurencyjnie przeliczalna, ale nie rozstrzygalna.
\end{uw}

\begin{lem}[przekątniowy Gödla; pierwszy raz sformułowany przez Carnapa; znany również jako lemat o punkcie stałym]
	Niech $\phi(x)$ to formuła z jedną zmienną wolną $x$.
	Wtedy istnieje zdanie $\psi$ takie, że $\PA^- \vdash \psi \iff \phi(\underline{\gnum{\psi}})$.
	Ponadto jeśli $\phi$ jest klasy $\Sigma_n$ lub $\Pi_n$, $n >0$, to $\psi$ też, odpowiednio.
\end{lem}
\begin{proof}
	Niech $\phi(x)$ to dowolna formuła z jedną zmienną wolną $x$.
	Rozważmy $\zeta(x) := \phi(\text{subst}(x, \text{num}(x))$.
	Niech $m := \gnum{\zeta}$, i niech $\psi := \zeta(\underline{m})$.
	Zauważmy, że $\gnum{\psi} = \gnum{\phi(\text{subst}(m, \text{num}(m))} = \text{subst}(m, \text{num}(m))$.
	Zatem $\PA^- \vdash \underline{\gnum{\psi}} = \text{subst}(m, \text{num}(m)$.
	A zatem
	\[
		\PA^- \vdash \psi \iff \phi(\text{subst}(m, \text{num}(m)) \iff \phi(\underline{\gnum{\psi}}).
	\]
\end{proof}

\begin{tw}[Tarskiego o niedefiniowalności prawdy]
	Niech $T \supseteq \PA^-$. (Wystarczy, że $T$ interpretuje $\PA^-$).
	Niech $\A \models T$.
	\underline{Nie} istnieje formuła $\emph{\text{Tr}}(x)$ taka, że dla dowolnego zdania $\psi$ zachodzi
	\[
		\A \models \psi \iff \emph{\text{Tr}}(\underline{\gnum{\psi}}).
	\]
\end{tw}
\begin{proof}
	Przypuśćmy przeciwnie -- niech $\text{Tr}(x)$ będzie taką właśnie formułą.
	Aplikując lemat przekątniowy do $\neg \text{Tr}(x)$, bierzemy $\psi$ takie, że
	\[
		 \A \models \psi \iff \neg \text{Tr}(\underline{\gnum{\psi}}).
	\]
	Ale przecież z założenia
	\[
		 \A \models \psi \iff \text{Tr}(\underline{\gnum{\psi}}),
	\]
	tak więc sprzeczność.
\end{proof}

\begin{df}
	Teoria $T \supseteq \PA^-$ jest $\omega$-\textit{niesprzeczna}, jeśli \underline{nie} istnieje formuła $\phi(x)$ taka, że
	\[
		 T \vdash \neg \phi(\underline{0}),\, T \vdash \neg \phi(\underline{1}),\, T \vdash \neg \phi(\underline{2}), \ldots
	\]
	i jednocześnie
	\[
		 T \vdash \exists x \, \phi(x).
	\]
\end{df}

\begin{tw}[I Gödla o niezupełności]
	 Niech $T$ bedzie teorią, która:
	 \begin{itemize}
		 \item zawiera (bądź interpretuje) $\PA^-$,
		 \item ma rozstrzygalny zbiór aksjomatów.
	 \end{itemize}
Wtedy istnieje zdanie $\gamma$ takie, że
\begin{itemize}
	\item jeśli $T$ jest niesprzeczna, to $T \not \vdash \gamma$,
	\item jeśli $T$ jest też $\omega$-niesprzeczna, to $T \not \vdash \neg \gamma$.
\end{itemize}
W szczególności, jeśli $T$ jest $\omega$-niesprzeczna, to jest niezupełna.
\end{tw}
% TODO tu ogólnie chyba trzeba jednak napisać o słabej/silnej reprezentowalności, i już nie mówić o Sigma_1-zupełności,bo jest trochę niejasno.
\begin{proof}
	Ponieważ $T$ ma rozstrzygalny zbiór aksjomatów, dysponujemy $\Sigma_1$-formułą $\text{Pr}_T(x)$.
	Korzystając z lematu przekątniowego, bierzemy zdanie $\gamma$ takie, że
	\begin{equation}
	\tag{*}
		 \PA^- \vdash \gamma \iff \neg \text{Pr}(\underline{\gnum{\gamma}}).
	\end{equation}

	Załóżmy, że $T \vdash \gamma$.
	Wtedy z (*) wynika, że $T \vdash \neg \text{Pr}_T(\underline{\gnum{\gamma}})$.
	Z drugiej strony, na mocy $\Sigma_1$-zupełności $T$, zachodzi $T \vdash \text{Pr}_T(\underline{\gnum{\gamma}})$.
	Czyli $T$ jest sprzeczna.
	W szczególności, zauważmy, że $\N \models \gamma$.

	Załóżmy teraz, że $T \vdash \neg \gamma$.
	W takim razie z (*) $T \vdash \exists x \, \text{Prov}_T(\underline{\gnum{\gamma}}, x)$.
	Ale
	\begin{gather*}
		T \vdash \neg \text{Prov}_T(\underline{\gnum{\gamma}}, \underline{0}), \\
		T \vdash \neg \text{Prov}_T(\underline{\gnum{\gamma}}, \underline{1}), \\
		T \vdash \neg \text{Prov}_T(\underline{\gnum{\gamma}}, \underline{2}), \\
		\vdots
	\end{gather*}
	bo żaden z ciągów znaków o numerach Gödla 1, 2, 3, \ldots nie jest dowodem $\gamma$ w $T$, a $T$ jest $\Sigma_1$-zupełna.
	Więc $T$ jest $\omega$-sprzeczna.
\end{proof}

\begin{uw}
	 Roser zrobił poprawkę do powyższego twierdzenia eliminując potrzebę $\omega$-niesprzeczności na rzecz zwykłej niesprzeczności.
\end{uw}

\begin{uw}
	 Niezupełność $\PA^-$ wynika w istocie z dyskretności modelu tej teorii, jej $\Sigma_1$-zupełności.
	 Modele typu $\RCF$ nie mają dyskretnych fragmentów i są zupełne.
\end{uw}

Teraz pokażemy II twierdzenie Gödla, mówiące o tym, że $\PA$ nie jest w stanie pokazać własnej niesprzeczności.

\begin{df}
	 Dla $T \supseteq \PA^-$ definiujemy zdanie ,,$T$ jest niesprzeczna'':
	 \[
	 	 \text{Con}_T := \neg \text{Pr}_T(\underline{\gnum{0 \neq 0}}).
	 \]
	(Zdanie to należy do klasy $\Pi_1$.)
\end{df}
% TODO usuń te kombinacje \emph{\text{}}
\begin{tw}[II Gödla o niezupełności]
	Jeśli teoria $T$ jest niesprzeczna i zawiera, bądź interpretuje $\PA$, to $T \not \vdash \emph{\text{Con}}_T$.
\end{tw}
\begin{uw}
	Zamiast $\PA$ wystarczyłoby $\mathsf{I}\Delta_0 + \text{exp}$ (tj. indukcja dla $\Delta_0$-formuł $+ \forall x \, y \, \exists z \, x^y = z$).
	Ale $\PA^-$ już nie wystarczy.
\end{uw}
\begin{proof}
	Niech $\gamma$ będzie zdaniem takim, jak w I twierdzeniu Gödla.
	Pokażemy, że $T \vdash \text{Con}_T \Rightarrow \gamma$.
	W tym celu będziemy korzystali z tego, że formuła $\text{Pr}_T(x)$ ma następujące własności (znane jako warunki Hilberta--Bernaysa--Löba):

	\mbox{  } \textbf{(D1)}:
	\[
		\text{jeśli } T \vdash \psi,\text{ to } \PA^- \vdash \text{Pr}_T(\underline{\gnum{\psi}}),
	\]

	\mbox{  } \textbf{(D2)}:
	\[
		T \vdash \text{Pr}_T(\underline{\gnum{\psi}}) \Rightarrow \text{Pr}_{\PA^-}(\underline{\gnum{\text{Pr}_T(\underline{\gnum{\psi}})}})
	\]

	\mbox{  } \textbf{(D3)}:
	\[
		T \vdash \text{Pr}_T(\underline{\gnum{\psi}}) \wedge \text{Pr}_T(\underline{\gnum{\psi \Rightarrow \phi}}) \Rightarrow \text{Pr}_T(\underline{\gnum{\phi}}).
	\]
	\textbf{(D1)} to jest po prostu $\Sigma_1$-zupełność $\PA^-$.
	\textbf{(D2)} wymaga uzasadnienia, że $T$ ,,wie'' o $\Sigma_1$-zupełności $\PA^-$ (sprawdzimy potem).
	\textbf{(D3)} wynika z możliwości konkatenacji dowodów.

	Pokażmy więc, że $T \vdash \neg \gamma \Rightarrow \neg \text{Con}_T$.

	Z jednej strony $T \vdash \neg \gamma \Rightarrow \text{Pr}_T(\underline{\gnum{\gamma}})$, a więc z \textbf{(D2)}:
	\[
		\tag{*}
		 T \vdash \neg \gamma \Rightarrow \text{Pr}_T(\underline{\gnum{\text{Pr}_T(\underline{\gnum{\gamma}})}}).
	\]
	Z drugiej zaś strony $T \vdash \gamma \Rightarrow \neg \text{Pr}_T(\underline{\gnum{\gamma}})$, a więc z \textbf{(D1)}:
	\[
		 T \vdash \text{Pr}_T(\underline{\gnum{\gamma \Rightarrow \neg \text{Pr}_T(\underline{\gnum{\gamma}})}}).
	\]
	A zatem z \textbf{(D3)}:
	\[
		 T \vdash \text{Pr}_T(\underline{\gnum{\gamma}}) \Rightarrow \text{Pr}_T(\underline{\gnum{\neg \text{Pr}_T(\underline{\gnum{\gamma}})}}).
	\]
	A więc, ponieważ $T \vdash \neg \gamma \Rightarrow \text{Pr}_T(\underline{\gnum{\gamma}})$, mamy:
	\[
		\tag{**}
		 T \vdash \neg \gamma \Rightarrow \text{Pr}_T(\underline{\gnum{\neg \text{Pr}_T(\underline{\gnum{\gamma}})}})
	\]
	Z (*, **) mamy:
	\[
		 T \vdash \neg \gamma \Rightarrow \left(\text{Pr}_T(\underline{\gnum{\text{Pr}_T(\underline{\gnum{\gamma}})}}) \wedge \text{Pr}_T(\underline{\gnum{\neg \text{Pr}_T(\underline{\gnum{\gamma}})}})\right)
	\]
	Stąd:
	\[
		 T \vdash \neg \gamma \Rightarrow \left(\text{Pr}_T(\underline{\gnum{\neg\gamma}}) \wedge \text{Pr}_T(\underline{\gnum{\gamma}})\right)
	\]
	A ponieważ prawa strona powyższej implikacji jest równoważna $\neg \text{Con}_T$ (co wynika $+/-$ z \textbf{(D3)}), to ostatecznie osiągamy cel:
	\[
		 T \vdash \neg \gamma \Rightarrow \neg \text{Con}_T
	\]
\end{proof}

%TODO uwagi światopoglądowe

\section{Formuły uniwersalne i hierarchia arytmetyczna}
% Cel na dziś:
\begin{tw}
	Dla każdego $n \geq 1$ istnieje formuła
	$\text{\emph{Sat}}_{\Sigma_n}(x, v)$ klasy $\Sigma_n$ taka, że dla
	dowolnej formuły $\phi(\bar{y}) \in \Sigma_n$
	\[
		\PA \vdash \forall \bar{y} \, (\phi(\bar{y}) \iff
		\text{\emph{Sat}}_{\Sigma_n}(\underline{\gnum{\phi}}, \langle
		\bar{y} \rangle))
	\]

\end{tw}

\begin{lem}
	Istnieje formuła $\text{\emph{Sat}}_{\Delta_0}(x, v)$ klasy
	$\Delta_0(\text{\emph{exp}})$ taka, że dla dowolnej formuły
	$\phi(\bar{y}) \in \Delta_0$
	\[
		\PA \vdash \forall \bar{y} \, (\phi(\bar{y}) \iff
		\text{\emph{Sat}}_{\Delta_0}(\underline{\gnum{\phi}}, \langle
		\bar{y} \rangle))
	\]

\end{lem}
\begin{proof}
	Najpierw konstruujemy formułę $\text{val}(x, v) = w$, która ma wyrażać, że ,,wartość termu o numerze Gödla $x$ przy wartościowaniu danym przez $v$ wynosi $w$''.
\end{proof}

%TODO dokończyć, może na końcu; przy pomocy książki Kaya

\section{Twierdzenie Tennenbauma}
\begin{df}
	Struktura $\A = (\N, \oplus, \odot, \cleq)$ jest \textit{strukturą obliczalną} (vel \textit{rekurencyjną}), jeśli $\oplus, \odot$ i $\cleq$ są obliczalnymi relacjami w $\N$.
\end{df}


%\section{Twierdzenie Ramseya i Parisa-Harringtona}

%\section{Twierdzenie Matijasiewicza i 10. problem Hilberta}

\end{document}
